\chapter{Geometry in Euclidean Space}
\section{Vector and Linear Combination}

\section{Point, Line, Plane, and Beyond}

\section{Parametric Curves and Surfaces}

\chapter{Matrices and System of Linear Equations}

\chapter{Vector Space and Subspace}

\section{Definitions and Properties of Vector Space}
\subsection{$\mathbb{C}$}
\begin{exercise}
Show that for every $\alpha \in \mathbb{C}$, there exists a unique $\beta \in \mathbf{C}$ such that $\alpha + \beta = 0$.
\end{exercise}

\begin{solution}
Given any $\alpha \in \mathbb{C}$, we need to find a $\beta \in \mathbf{C}$ such that $\alpha + \beta = 0$. Let $\beta = -\alpha$. Then:
\[
\alpha + \beta = \alpha + (-\alpha) = 0.
\]
Thus, for every $\alpha \in \mathbb{C}$, $\beta = -\alpha$ satisfies the equation $\alpha + \beta = 0$. To show uniqueness, assume there exists another $\beta' \in \mathbf{C}$ such that $\alpha + \beta' = 0$. Then:
\[
\alpha + \beta = \alpha + \beta' = 0 \implies \beta = \beta'.
\]
Therefore, $\beta = -\alpha$ is the unique solution.
\end{solution}

\begin{exercise}
Show that for every $\alpha \in \mathbb{C}$ with $\alpha \neq 0$, there exists a unique $\beta \in \mathbb{C}$ such that $\alpha \beta = 1$.
\end{exercise}

\begin{solution}
Given any $\alpha \in \mathbf{C}$ with $\alpha \neq 0$, we need to find a $\beta \in \mathbb{C}$ such that $\alpha \beta = 1$. Let $\beta = \frac{1}{\alpha}$. Then:
\[
\alpha \beta = \alpha \left( \frac{1}{\alpha} \right) = 1.
\]
Thus, for every $\alpha \in \mathbb{C}$ with $\alpha \neq 0$, $\beta = \frac{1}{\alpha}$ satisfies the equation $\alpha \beta = 1$. To show uniqueness, assume there exists another $\beta' \in \mathbb{C}$ such that $\alpha \beta' = 1$. Then:
\[
\alpha \beta = \alpha \beta' = 1 \implies \beta = \beta'.
\]
Therefore, $\beta = \frac{1}{\alpha}$ is the unique solution.
\end{solution}

\begin{exercise}
Show that $(ab)x = a(bx)$ for all $x \in \mathbf{F}^{n}$ and all $a, b \in \mathbf{F}$.
\end{exercise}

\begin{solution}
Let $x \in \mathbf{F}^{n}$ and $a, b \in \mathbf{F}$. We want to show that $(ab)x = a(bx)$. 

First, recall that in a vector space over a field $\mathbf{F}$, the scalar multiplication is associative. This means that for any scalar $c \in \mathbf{F}$ and vector $y \in \mathbf{F}^{n}$, we have $c(y) = (cy)$. 

Now, consider the left-hand side of the equation:
\[
(ab)x
\]
By the definition of scalar multiplication in a vector space, multiplying a vector by a scalar is the same as multiplying each component of the vector by the scalar. Hence, we can write:
\[
(ab)x = (ab) \cdot x
\]

Next, consider the right-hand side of the equation:
\[
a(bx)
\]
First, compute $bx$:
\[
bx = b \cdot x
\]
Then, multiplying the result by $a$ gives:
\[
a(bx) = a \cdot (b \cdot x)
\]

By the associativity of scalar multiplication, we have:
\[
a \cdot (b \cdot x) = (a \cdot b) \cdot x = (ab)x
\]

Thus, we have shown that:
\[
(ab)x = a(bx)
\]
for all $x \in \mathbf{F}^{n}$ and all $a, b \in \mathbf{F}$.
\end{solution}

\begin{exercise}
Show that $(a+b)x = ax + bx$ for all $a, b \in \mathbf{F}$ and all $x \in \mathbf{F}^{n}$.
\end{exercise}

\begin{solution}
Let $x \in \mathbf{F}^{n}$ and $a, b \in \mathbf{F}$. We want to show that $(a+b)x = ax + bx$.

Recall that in a vector space over a field $\mathbf{F}$, scalar multiplication distributes over vector addition. This means that for any scalars $c, d \in \mathbf{F}$ and any vector $y \in \mathbf{F}^{n}$, we have:
\[
(c + d)y = cy + dy.
\]

Now, consider the left-hand side of the equation:
\[
(a + b)x.
\]
By the definition of scalar multiplication in a vector space, we have:
\[
(a + b)x = (a + b) \cdot x.
\]

Next, consider the right-hand side of the equation:
\[
ax + bx.
\]
First, compute $ax$ and $bx$:
\[
ax = a \cdot x,
\]
\[
bx = b \cdot x.
\]
Then, add these results together:
\[
ax + bx = a \cdot x + b \cdot x.
\]

By the distributive property of scalar multiplication over vector addition, we have:
\[
(a + b) \cdot x = a \cdot x + b \cdot x.
\]

Thus, we have shown that:
\[
(a + b)x = ax + bx
\]
for all $a, b \in \mathbf{F}$ and all $x \in \mathbf{F}^{n}$.
\end{solution}

\begin{exercise}
Show that $\lambda(x + y) = \lambda x + \lambda y$ for all $\lambda \in \mathbf{F}$ and all $x, y \in \mathbf{F}^{n}$.
\end{exercise}

\begin{solution}
Let $x, y \in \mathbf{F}^{n}$ and $\lambda \in \mathbf{F}$. We want to show that $\lambda(x + y) = \lambda x + \lambda y$.

Recall that in a vector space over a field $\mathbf{F}$, scalar multiplication distributes over vector addition. This means that for any scalar $\lambda \in \mathbf{F}$ and any vectors $x, y \in \mathbf{F}^{n}$, we have:
\[
\lambda(x + y) = \lambda x + \lambda y.
\]

Now, consider the left-hand side of the equation:
\[
\lambda(x + y).
\]
By the definition of vector addition and scalar multiplication in a vector space, we have:
\[
x + y \in \mathbf{F}^{n}.
\]
Then, scalar multiplication distributes over this vector addition:
\[
\lambda(x + y).
\]

Next, consider the right-hand side of the equation:
\[
\lambda x + \lambda y.
\]
First, compute $\lambda x$ and $\lambda y$:
\[
\lambda x \in \mathbf{F}^{n},
\]
\[
\lambda y \in \mathbf{F}^{n}.
\]
Then, add these results together:
\[
\lambda x + \lambda y \in \mathbf{F}^{n}.
\]

By the distributive property of scalar multiplication over vector addition, we have:
\[
\lambda(x + y) = \lambda x + \lambda y.
\]

Thus, we have shown that:
\[
\lambda(x + y) = \lambda x + \lambda y
\]
for all $\lambda \in \mathbf{F}$ and all $x, y \in \mathbf{F}^{n}$.
\end{solution}

\section{$F^n$ and $F^S$}
\begin{definition}[$F^n$]
Let $\mathbf{F}$ be a field and $n$ be a positive integer. The set $\mathbf{F}^n$ is defined as the set of all $n$-tuples of elements from $\mathbf{F}$:
\[
\mathbf{F}^n = \{ (x_1, x_2, \ldots, x_n) \mid x_i \in \mathbf{F} \text{ for } i = 1, 2, \ldots, n \}.
\]
Elements of $\mathbf{F}^n$ are called vectors, and $\mathbf{F}^n$ is called an $n$-dimensional vector space over the field $\mathbf{F}$.
\end{definition}
\begin{definition}[$F^S$]
Let $\mathbf{F}$ be a field and $S$ be a set. The set $\mathbf{F}^S$ is defined as the set of all functions from $S$ to $\mathbf{F}$:
\[
\mathbf{F}^S = \{ f \mid f: S \to \mathbf{F} \}.
\]
Each function $f \in \mathbf{F}^S$ assigns an element of $\mathbf{F}$ to each element of $S$. $\mathbf{F}^S$ is called a vector space over the field $\mathbf{F}$.
\end{definition}
\begin{exercise}
Suppose $a \in \mathbf{F}, v \in V$, and $av = 0$. Prove that $a = 0$ or $v = 0$. 
\end{exercise}

\begin{solution}
We will prove this statement by contradiction.

Assume the contrary, that is, suppose $a \neq 0$ and $v \neq 0$. 

Since $a \neq 0$, it has a multiplicative inverse $a^{-1}$ in the field $\mathbf{F}$. 

Now, consider the vector $a^{-1}(av)$. Using the associativity of scalar multiplication, we have:
\[
a^{-1}(av) = (a^{-1}a)v = 1v = v.
\]

Given that $av = 0$, we substitute this into the above expression:
\[
a^{-1}(0) = 0.
\]

So, we have:
\[
v = 0,
\]
which contradicts our assumption that $v \neq 0$.

Therefore, our assumption that both $a \neq 0$ and $v \neq 0$ must be false. Hence, it must be that either $a = 0$ or $v = 0$.
\end{solution}

\begin{exercise}
Prove that $-(-v) = v$ for every $v \in V$.
\end{exercise}

\begin{solution}
By definition of the additive inverse, we have:
\[
(-v) + (-(-v)) = 0 \quad \text{and} \quad v + (-v) = 0.
\]

This implies that both $v$ and $-(-v)$ are additive inverses of $-v$. By the uniqueness of the additive inverse, it follows that:
\[
-(-v) = v.
\]
\end{solution}

\begin{exercise}
Suppose $v, w \in V$. Explain why there exists a unique $x \in V$ such that $v + 3x = w$.
\end{exercise}

\begin{solution}
To find a unique $x \in V$ such that $v + 3x = w$, we proceed as follows:

First, isolate $3x$ by subtracting $v$ from both sides of the equation:
\[
v + 3x = w \implies 3x = w - v.
\]

Since $3 \neq 0$ in the field $\mathbf{F}$, there exists a multiplicative inverse of $3$, denoted as $\frac{1}{3}$. Multiply both sides of the equation $3x = w - v$ by $\frac{1}{3}$:
\[
x = \frac{1}{3} (w - v).
\]

We have now found an $x \in V$ that satisfies the equation. To prove the uniqueness of $x$, suppose there exists another $x' \in V$ such that $v + 3x' = w$. Then:
\[
3x' = w - v.
\]

Multiplying both sides by $\frac{1}{3}$, we obtain:
\[
x' = \frac{1}{3} (w - v).
\]

Thus, $x = x'$, showing that the solution $x$ is unique.

Therefore, for any $v, w \in V$, there exists a unique $x \in V$ such that $v + 3x = w$.
\end{solution}

\begin{exercise}
Show that in the definition of a vector space, the additive inverse condition can be replaced with the condition that
\[
0 v = 0 \quad \text{for all } v \in V,
\]
where the 0 on the left side is the scalar zero, and the 0 on the right side is the additive identity of $V$.
\end{exercise}

\begin{solution}
We begin by noting the properties of scalar multiplication and the existence of the zero vector in a vector space. The condition $0v = 0_V$ for all $v \in V$ implies that multiplying any vector by the scalar zero yields the zero vector, consistent with vector space axioms.

To demonstrate the additive inverse property using $0v = 0_V$, consider:
\[
0v = (0+0)v = 0v + 0v.
\]
Subtracting $0v$ from both sides, we have:
\[
0v = 0_V.
\]

We leverage the distributive property of scalar multiplication over vector addition to find an additive inverse:
\[
1v + (-1)v = (1-1)v = 0v = 0_V.
\]
This simplifies to:
\[
v + (-1)v = 0_V,
\]
where $(-1)v$ acts as the additive inverse of $v$. Thus, $(-1)v = -v$, and we confirm:
\[
v + (-v) = 0_V.
\]

Thus, the condition $0v = 0_V$ indeed implies the existence of an additive inverse for every vector $v$ in $V$, allowing us to replace the explicit requirement for an additive inverse with the condition $0v = 0_V$ in the definition of a vector space.
\end{solution}

\begin{exercise}
Let $\infty$ and $-\infty$ denote two distinct objects, neither of which is in $\mathbf{R}$. Define an addition and scalar multiplication on $\mathbf{R} \cup\{\infty,-\infty\}$ as you could guess from the notation. Specifically, the sum and product of two real numbers is as usual, and for $t \in \mathbf{R}$ define

\[
t \infty=\left\{\begin{array}{ll}
-\infty & \text { if } t<0 \\
0 & \text { if } t=0, \\
\infty & \text { if } t>0
\end{array} \quad t(-\infty)= \begin{cases}\infty & \text { if } t<0 \\
0 & \text { if } t=0 \\
-\infty & \text { if } t>0\end{cases}\right.
\]

and

\[
\begin{aligned}
t+\infty & =\infty+t=\infty+\infty=\infty \\
t+(-\infty) & =(-\infty)+t=(-\infty)+(-\infty)=-\infty \\
\infty+(-\infty) & =(-\infty)+\infty=0
\end{aligned}
\]

With these operations of addition and scalar multiplication, is $\mathbf{R} \cup\{\infty,-\infty\}$ a vector space over $\mathbf{R}$ ? Explain.
\end{exercise}
\begin{solution}
This is not a vector space over $\mathbb{R}$. Consider the distributive properties. If this is a vector space over $\mathbb{R}$, we will have
$$
\infty=(2+(-1)) \infty=2 \infty+(-1) \infty=\infty+(-\infty)=0 \text {. }
$$

Hence for any $t \in \mathbb{R}$, one has
$$
t=0+t=\infty+t=\infty=0 .
$$

We get a contradiction since zero vector is unique.
\end{solution}

\begin{exercise}
Suppose $V$ is a real vector space.
The complexification of $V$, denoted by $V_{\mathrm{C}}$, equals $V \times V$. An element of $V_{\mathrm{C}}$ is an ordered pair $(u, v)$, where $u, v \in V$, but we write this as $u+i v$.
Addition on $V_{\mathrm{C}}$ is defined by

\[
\left(u_{1}+i v_{1}\right)+\left(u_{2}+i v_{2}\right)=\left(u_{1}+u_{2}\right)+i\left(v_{1}+v_{2}\right)
\]

for all $u_{1}, v_{1}, u_{2}, v_{2} \in V$.
- Complex scalar multiplication on $V_{\mathrm{C}}$ is defined by

\[
(a+b i)(u+i v)=(a u-b v)+i(a v+b u)
\]

for all $a, b \in \mathbf{R}$ and all $u, v \in V$.

Prove that with the definitions of addition and scalar multiplication as above, $V_{\mathrm{C}}$ is a complex vector space.

Think of $V$ as a subset of $V_{\mathrm{C}}$ by identifying $u \in V$ with $u+i 0$. The construction of $V_{\mathrm{C}}$ from $V$ can then be thought of as generalizing the construction of $\mathbf{C}^{n}$ from $\mathbf{R}^{n}$.
\end{exercise}

\begin{solution}
To prove that \( V_C \) is closed under complex scalar multiplication, we need to show that for any complex scalar \( a + bi \in \mathbb{C} \) and any element \( u + iv \in V_C \), the product \( (a + bi)(u + iv) \) is also in \( V_C \).

Recall the definition of complex scalar multiplication on \( V_C \):
\[ (a + bi)(u + iv) = (au - bv) + i(av + bu) \]

We need to verify that the result of this multiplication, \( (au - bv) + i(av + bu) \), is an element of \( V_C \).

\begin{itemize}
    \item The real part of the product is \( au - bv \).
    \item The imaginary part of the product is \( av + bu \).
\end{itemize}

Since \( u, v \in V \) and \( V \) is a real vector space, we know the following:
\begin{itemize}
    \item \( u, v \in V \implies au, bv \in V \) because \( V \) is closed under scalar multiplication.
    \item \( V \) is closed under addition, so \( au - bv \in V \).
    \item Similarly, \( av, bu \in V \) and \( V \) is closed under addition, so \( av + bu \in V \).
\end{itemize}

Therefore, both \( au - bv \in V \) and \( av + bu \in V \), implying that \( (au - bv) + i(av + bu) \in V_C \).

Thus, we have shown that \( (a + bi)(u + iv) \in V_C \), confirming that \( V_C \) is closed under complex scalar multiplication.

\end{solution}


\section{Subspace of Vector Space}
\begin{definition}[Subspace]
    A subset $U$ of $V$ is called a subspace of $V$ if $U$ is also a vector space with the same additive identity, addition, and scalar multiplication as on $V$.
\end{definition}
\subsection{Criteria of Subspace}
\begin{proposition}[Conditions for a Subspace]
    A subset $U$ of $V$ is a subspace of $V$ if and only if $U$ satisfies the following three conditions.

\begin{enumerate}
    \item \textbf{additive identity}: $0 \in U$.
    \item \textbf{closed under addition}: $u, w \in U$ implies $u+w \in U$.
    \item \textbf{closed under scalar multiplication}: $a \in \mathbb{F}$ and $u \in U$ implies $a u \in U$.
\end{enumerate}

\end{proposition}

\begin{exercise}
1 For each of the following subsets of $\mathbb{F}^{3}$, determine whether it is a subspace of $\mathbb{F}^{3}$.

(a) $\left\{\left(x_{1}, x_{2}, x_{3}\right) \in \mathbb{F}^{3}: x_{1}+2 x_{2}+3 x_{3}=0\right\}$

(b) $\left\{\left(x_{1}, x_{2}, x_{3}\right) \in \mathbb{F}^{3}: x_{1}+2 x_{2}+3 x_{3}=4\right\}$

(c) $\left\{\left(x_{1}, x_{2}, x_{3}\right) \in \mathbb{F}^{3}: x_{1} x_{2} x_{3}=0\right\}$

(d) $\left\{\left(x_{1}, x_{2}, x_{3}\right) \in \mathbb{F}^{3}: x_{1}=5 x_{3}\right\}$
\end{exercise}
\begin{solution}
We apply the rules given.
\begin{enumerate}
    \item[(a)] \(\left\{ \left( x_{1}, x_{2}, x_{3} \right) \mid x_{1} + 2x_{2} + 3x_{3} = 0 \right\}\)
    \begin{enumerate}
        \item \textbf{Additive Identity:} \(0 + 2 \cdot 0 + 3 \cdot 0 = 0\), so \( \mathbf{0} \in U \).
        \item \textbf{Closed under Addition:} If \(u = (u_{1}, u_{2}, u_{3}) \in U\) and \(w = (w_{1}, w_{2}, w_{3}) \in U\), then \((u_{1} + w_{1}) + 2(u_{2} + w_{2}) + 3(u_{3} + w_{3}) = 0\), so \(u + w \in U\).
        \item \textbf{Closed under Scalar Multiplication:} If \(u = (u_{1}, u_{2}, u_{3}) \in U\) and \(a \in \mathbb{F}\), then \(a(u_{1} + 2u_{2} + 3u_{3}) = 0\), so \(au \in U\).
    \end{enumerate}
    Hence, \( U \) is a subspace of \(\mathbb{F}^{3}\).

    \item[(b)] \(\left\{ \left( x_{1}, x_{2}, x_{3} \right) \mid x_{1} + 2x_{2} + 3x_{3} = 4 \right\}\)
    \begin{enumerate}
        \item \textbf{Additive Identity:} \(0 + 2 \cdot 0 + 3 \cdot 0 = 0 \neq 4\), so \( \mathbf{0} \notin U \).
    \end{enumerate}
    Hence, \( U \) is not a subspace of \(\mathbb{F}^{3}\).

    \item[(c)] \(\left\{ \left( x_{1}, x_{2}, x_{3} \right) \mid x_{1} x_{2} x_{3} = 0 \right\}\)
    \begin{enumerate}
        \item \textbf{Additive Identity:} \(0 \cdot 0 \cdot 0 = 0\), so \( \mathbf{0} \in U \).
        \item \textbf{Closed under Addition:} \(u = (1, 0, 0) \in U\) and \(w = (0, 1, 0) \in U\), but \(u + w = (1, 1, 0) \notin U\).
    \end{enumerate}
    Hence, \( U \) is not a subspace of \(\mathbb{F}^{3}\).

    \item[(d)] \(\left\{ \left( x_{1}, x_{2}, x_{3} \right) \mid x_{1} = 5x_{3} \right\}\)
    \begin{enumerate}
        \item \textbf{Additive Identity:} \(0 = 5 \cdot 0\), so \( \mathbf{0} \in U \).
        \item \textbf{Closed under Addition:} If \(u = (5u_{3}, u_{2}, u_{3}) \in U\) and \(w = (5w_{3}, w_{2}, w_{3}) \in U\), then \(u + w = (5(u_{3} + w_{3}), u_{2} + w_{2}, u_{3} + w_{3}) \in U\).
        \item \textbf{Closed under Scalar Multiplication:} If \(u = (5u_{3}, u_{2}, u_{3}) \in U\) and \(a \in \mathbb{F}\), then \(au = (5(au_{3}), au_{2}, au_{3}) \in U\).
    \end{enumerate}
    Hence, \( U \) is a subspace of \(\mathbb{F}^{3}\).
\end{enumerate}
\end{solution}

\begin{exercise}
Justify each of the following sets is a subspace, providing necessary conditions where applicable.

\begin{enumerate}
    \item[(a)] If \( b \in \mathbb{F} \), then \(\left\{(x_{1}, x_{2}, x_{3}, x_{4}) \in \mathbb{F}^{4} \mid x_{3} = 5x_{4} + b \right\}\) is a subspace of \(\mathbb{F}^{4}\) if and only if \( b = 0 \).
    
    \item[(b)] The set of continuous real-valued functions on the interval \([0,1]\) is a subspace of \(\mathbb{R}^{[0,1]}\).
    
    \item[(c)] The set of differentiable real-valued functions on \(\mathbb{R}\) is a subspace of \(\mathbb{R}^{\mathbb{R}}\).
    
    \item[(d)] The set of differentiable real-valued functions \( f \) on the interval \((0,3)\) such that \( f'(2) = b \) is a subspace of \(\mathbb{R}^{(0,3)}\) if and only if \( b = 0 \).
    
    \item[(e)] The set of all sequences of complex numbers with limit \( 0 \) is a subspace of \(\mathbb{C}^{\infty}\).
\end{enumerate}
\end{exercise}
\begin{solution}
    By conditions of a subspace.
    \begin{enumerate}
        \item[(a)] Let \( x_1 = m \), \( x_2 = n \), \( x_4 = a \), \( x_3 = 5a + b \). Since \( b \in \mathbb{F} \) and \( x_1, x_2, x_4 \in \mathbb{F} \), \( x_3 \in \mathbb{F} \). Thus we can show that the subspace \( U \) is closed under multiplication and addition. Now we only need to show the existence of additive identity. We only need to show that \( x_3 \) can be 0, since the rest of the components are all in \( \mathbb{F} \). As \( b \in \mathbb{F} \), when \( b = 0 \) and \( x_4 = 0 \), \( x_3 = 0 \), and thus \( \mathbf{0} \in \mathbb{F}^4 \). Hence, \( U \) is a subspace of \( \mathbb{F}^4 \) when \( b = 0 \).
        
        \item[(b)] The set of continuous real-valued functions on the interval \([0,1]\), denoted by \( C([0,1]) \), is a subspace of \( \mathbb{R}^{[0,1]} \). It is obvious that the zero function \( f(x) = 0 \) for all \( x \in [0,1] \) is continuous and hence belongs to \( C([0,1]) \). Also, if \( f \) and \( g \) are continuous functions on \([0,1]\), then \( f + g \) and \( cf \) (for any scalar \( c \in \mathbb{R} \)) are also continuous on \([0,1]\). Therefore, \( C([0,1]) \) is closed under addition and scalar multiplication, making it a subspace.
        
        \item[(c)] The set of differentiable real-valued functions on \( \mathbb{R} \), denoted by \( D(\mathbb{R}) \), is a subspace of \( \mathbb{R}^{\mathbb{R}} \). The zero function \( f(x) = 0 \) for all \( x \in \mathbb{R} \) is differentiable. If \( f \) and \( g \) are differentiable functions, then their sum \( f + g \) and scalar multiple \( cf \) are also differentiable. Therefore, \( D(\mathbb{R}) \) is closed under addition and scalar multiplication, making it a subspace.
        
        \item[(d)] The set of differentiable real-valued functions \( f \) on the interval \( (0,3) \) such that \( f'(2) = b \), denoted by \( D_b((0,3)) \), is a subspace of \( \mathbb{R}^{(0,3)} \) if and only if \( b = 0 \). To be a subspace, it must include the zero function, which means \( f'(2) = 0 \) for the zero function. This implies \( b = 0 \). For \( b = 0 \), the set \( D_0((0,3)) \) includes the zero function. Additionally, if \( f \) and \( g \) are in \( D_0((0,3)) \), then \( (f + g)'(2) = f'(2) + g'(2) = 0 + 0 = 0 \), and if \( f \) is in \( D_0((0,3)) \) and \( c \) is a scalar, then \( (cf)'(2) = cf'(2) = c \cdot 0 = 0 \). Thus, \( D_0((0,3)) \) is closed under addition and scalar multiplication, making it a subspace.
        
        \item[(e)] The set of all sequences of complex numbers with limit \( 0 \) is a subspace of \( \mathbb{C}^{\infty} \). Let \( S \) be the set of all sequences of complex numbers with limit \( 0 \). The zero sequence \( (0, 0, 0, \ldots) \) is in \( S \). If \( (a_n) \) and \( (b_n) \) are in \( S \), then \( \lim_{n \to \infty} a_n = 0 \) and \( \lim_{n \to \infty} b_n = 0 \). Thus, \( \lim_{n \to \infty} (a_n + b_n) = \lim_{n \to \infty} a_n + \lim_{n \to \infty} b_n = 0 + 0 = 0 \), so \( (a_n + b_n) \in S \). If \( (a_n) \in S \) and \( c \in \mathbb{C} \), then \( \lim_{n \to \infty} (c a_n) = c \lim_{n \to \infty} a_n = c \cdot 0 = 0 \), so \( (c a_n) \in S \). Thus, \( S \) is closed under addition and scalar multiplication, making it a subspace.
    \end{enumerate}
\end{solution}

\begin{exercise}
    Show that the set of differentiable real-valued functions $f$ on the interval $(-4,4)$ such that $f^{\prime}(-1)=3 f(2)$ is a subspace of $\mathbf{R}^{(-4,4)}$.
\end{exercise}
\begin{solution}
To prove that the set \( V = \{ f : (-4,4) \to \mathbb{R} \mid f \text{ is differentiable and } f'(-1) = 3f(2) \} \) is a subspace of \( \mathbb{R}^{(-4,4)} \), we need to verify the following three conditions:

\begin{enumerate}
    \item \( V \) contains the zero vector (zero function).
    \item \( V \) is closed under addition.
    \item \( V \) is closed under scalar multiplication.
\end{enumerate}

\paragraph{1. Zero function:}

Consider the zero function \( f(x) = 0 \). We need to verify that it belongs to \( V \).

\[
f'(x) = 0 \text{ for all } x \in (-4,4)
\]

Specifically, \( f'(-1) = 0 \) and \( f(2) = 0 \). Therefore,

\[
f'(-1) = 0 = 3 \cdot 0 = 3f(2)
\]

Hence, the zero function belongs to \( V \).

\paragraph{2. Closed under addition:}

Assume \( f \) and \( g \) both belong to \( V \). We need to show that \( f + g \) also belongs to \( V \).

Since \( f \) and \( g \) belong to \( V \), we have

\[
f'(-1) = 3f(2) \quad \text{and} \quad g'(-1) = 3g(2)
\]

Consider \( h = f + g \). We need to check if \( h \) satisfies \( h'(-1) = 3h(2) \).

Since \( h = f + g \), we have

\[
h' = f' + g'
\]

Thus,

\[
h'(-1) = f'(-1) + g'(-1)
\]

and

\[
h(2) = f(2) + g(2)
\]

Therefore,

\[
h'(-1) = f'(-1) + g'(-1) = 3f(2) + 3g(2) = 3(f(2) + g(2)) = 3h(2)
\]

This shows that \( f + g \) also satisfies \( f'(-1) = 3f(2) \), so \( f + g \) belongs to \( V \).

\paragraph{3. Closed under scalar multiplication:}

Assume \( f \) belongs to \( V \) and \( c \) is a scalar. We need to show that \( cf \) also belongs to \( V \).

Since \( f \) belongs to \( V \), we have

\[
f'(-1) = 3f(2)
\]

Consider \( h = cf \). We need to check if \( h \) satisfies \( h'(-1) = 3h(2) \).

Since \( h = cf \), we have

\[
h' = c f'
\]

Thus,

\[
h'(-1) = c f'(-1)
\]

and

\[
h(2) = c f(2)
\]

Therefore,

\[
h'(-1) = c f'(-1) = c \cdot 3f(2) = 3c f(2) = 3h(2)
\]

This shows that \( cf \) also satisfies \( f'(-1) = 3f(2) \), so \( cf \) belongs to \( V \).
\end{solution}

\begin{exercise}
    Suppose $b \in \mathbf{R}$. Show that the set of continuous real-valued functions $f$ on the interval $[0,1]$ such that $\int_{0}^{1} f=b$ is a subspace of $\mathbf{R}^{[0,1]}$ if and only if $b=0$.
\end{exercise}

\begin{solution}
Denote the set of continuous real-valued functions $f$ on the interval $[0,1]$ such that $\int_0^1 f=b$ by $V_b$.

If $V_b$ is a subspace of $\mathbb{R}^{[0,1]}$, then for any $f \in V_b$, we have $\int_0^1 f=b$. Because $V_b$ is a subspace of $\mathbb{R}^n$, it follows that $k f \in V_b$ for any $k \in \mathbb{R}$. Hence
$$
b=\int_0^1(k f)=k \int_0^1 f=k b, \quad \text { for all } k \in \mathbb{R},
$$
this happens if and only if $b=0$.

Now if $b=0$, then for any $f, g \in V_0$ and $\lambda \in \mathbb{R}$. We have that
$$
\int_0^1(f+g)=\int_0^1 f+\int_0^1 g=0+0=0
$$
and $f+g$ is continuous real-valued functions since $f$ and $g$ are. This deduces $f+g \in V_0$, i.e. $V_0$ is closed under addition. Similarly,
$$
\int_0^1(\lambda f)=\lambda \int_0^1 f=k 0=0
$$
and $\lambda f$ is continuous real-valued functions since $f$ is. This implies $\lambda f \in V_0$, i.e. $V_0$ is closed under scalar multiplication. On the other hand, the constant function $f \equiv 0 \in V_0$, which is also the additive identity in $\mathbb{R}^{[0,1]}$. Hence $V_0$ is a subspace of $\mathbb{R}^n$ by 1.34 .
\end{solution}

\begin{exercise}
    A function $f: \mathbf{R} \rightarrow \mathbf{R}$ is called periodic if there exists a positive number $p$ such that $f(x)=f(x+p)$ for all $x \in \mathbf{R}$. Is the set of periodic functions from $\mathbf{R}$ to $\mathbf{R}$ a subspace of $\mathbf{R}^{\mathbf{R}}$ ? Explain.
\end{exercise}
\begin{solution}
Let's analyze whether the set of periodic functions from \(\mathbf{R}\) to \(\mathbf{R}\) forms a subspace of \(\mathbf{R}^{\mathbf{R}} \), the vector space of all functions from \(\mathbf{R}\) to \(\mathbf{R}\).

To be a subspace, the set of periodic functions must satisfy three conditions:

1. \textbf{Contain the zero vector:} The zero vector in \(\mathbf{R}^{\mathbf{R}}\) is the zero function \(f(x) = 0\) for all \(x \in \mathbf{R}\). This function is trivially periodic with any period \(p > 0\), so the zero function is in the set of periodic functions.

2. \textbf{Closed under addition:} Suppose \(f\) and \(g\) are periodic functions with periods \(p_f\) and \(p_g\), respectively. That means \(f(x) = f(x + p_f)\) and \(g(x) = g(x + p_g)\) for all \(x \in \mathbf{R}\). We need to determine if the function \(h = f + g\) is periodic. For \(h\) to be periodic, there must exist a period \(p > 0\) such that \(h(x) = h(x + p)\) for all \(x\). However, the sum of two periodic functions is not necessarily periodic unless their periods are commensurate (i.e., there exists some positive integer multiples of \(p_f\) and \(p_g\) that are equal). In general, \(f(x + p) + g(x + p)\) might not equal \(f(x) + g(x)\) unless \(p\) is a common multiple of \(p_f\) and \(p_g\), which is not guaranteed for arbitrary periodic functions. Hence, the set is not closed under addition.

3. \textbf{Closed under scalar multiplication:} Suppose \(f\) is a periodic function with period \(p\) and \(\lambda\) is a scalar. The function \(\lambda f\) is periodic with the same period \(p\) because \(\lambda f(x + p) = \lambda f(x)\). Hence, the set is closed under scalar multiplication.

Since the set of periodic functions is not closed under addition, it fails to meet one of the necessary conditions for being a subspace. Therefore, the set of periodic functions from \(\mathbf{R}\) to \(\mathbf{R}\) is \textbf{not} a subspace of \(\mathbf{R}^{\mathbf{R}}\).

In conclusion, the composition (addition) of two periodic functions does not necessarily result in a new periodic function unless the functions have the same period or commensurate periods, so the set is not closed under addition.
\end{solution}
