%%%%%%%%%%%%%%%%%%%%%%%%%%%%%%%%%%%%%%%%%
% The Legrand Orange Book
% LaTeX Template
% Version 3.1 (February 18, 2022)
%
% This template originates from:
% https://www.LaTeXTemplates.com
%
% Authors:
% Vel (vel@latextemplates.com)
% Mathias Legrand (legrand.mathias@gmail.com)
%
% License:
% CC BY-NC-SA 4.0 (https://creativecommons.org/licenses/by-nc-sa/4.0/)
%
% Compiling this template:
% This template uses biber for its bibliography and makeindex for its index.
% When you first open the template, compile it from the command line with the 
% commands below to make sure your LaTeX distribution is configured correctly:
%
% 1) pdflatex main
% 2) makeindex main.idx -s indexstyle.ist
% 3) biber main
% 4) pdflatex main x 2
%
% After this, when you wish to update the bibliography/index use the appropriate
% command above and make sure to compile with pdflatex several times 
% afterwards to propagate your changes to the document.
%
%%%%%%%%%%%%%%%%%%%%%%%%%%%%%%%%%%%%%%%%%

%----------------------------------------------------------------------------------------
%	PACKAGES AND OTHER DOCUMENT CONFIGURATIONS
%----------------------------------------------------------------------------------------

\documentclass[
	12pt, % Default font size, select one of 10pt, 11pt or 12pt
	fleqn, % Left align equations
	a4paper, % Paper size, use either 'a4paper' for A4 size or 'letterpaper' for US letter size
	%oneside, % Uncomment for oneside mode, this doesn't start new chapters and parts on odd pages (adding an empty page if required), this mode is more suitable if the book is to be read on a screen instead of printed
]{LegrandOrangeBook}

% Book information for PDF metadata, remove/comment this block if not required 
\hypersetup{
	pdftitle={Title}, % Title field
	pdfauthor={Author}, % Author field
	pdfsubject={Subject}, % Subject field
	pdfkeywords={Keyword1, Keyword2, ...}, % Keywords
	pdfcreator={LaTeX}, % Content creator field
}
\usepackage{booktabs}
\usepackage[table]{xcolor}
\usepackage{algorithm}
%\usepackage{enumitem}
\usepackage{algpseudocode}
\usepackage{wrapfig}
\usepackage{lipsum}
\usepackage{karnaugh-map}
\usepackage{amsthm}
\usepackage{thmtools}
\usepackage{amsmath}
\usepackage{etoolbox}
\usepackage{tikz}
\usepackage{pgfplots}
\usepackage{pgfplotstable}
\usetikzlibrary{automata}
\usetikzlibrary{arrows.meta}
\usepackage{comment}
\usepackage{array}
\usepackage{booktabs}
\usetikzlibrary{3d, shapes.geometric, positioning}

%%%%%%%%%%%%%%%%%%%%%%%%%%%%%%%%%%%%%%%%%%%%%%%%%%%%%%%%%%%%%%%%%%%%%%%%%%%%%%%%%%%%%%%
% Shorthand
\newcommand{\vect}[1]{\mathbf{#1}} % For regular vectors
\newcommand{\uvec}[1]{\hat{\mathbf{#1}}} % For unit vectors

\newcommand{\R}{\mathbb{R}} % Real numbers
\newcommand{\Z}{\mathbb{Z}} % Integers
\newcommand{\C}{\mathbb{C}} % Complex numbers
\newcommand{\N}{\mathbb{N}} % Natural numbers
\newcommand{\Q}{\mathbb{Q}} % Rational numbers
\newcommand{\Hq}{\mathbb{H}} % Quaternions
\newcommand{\F}{\mathbb{F}} % Finite fields
\newcommand{\Proj}{\mathbb{P}} % Projective space
\newcommand{\K}{\mathbb{K}} % Arbitrary field
\newcommand{\T}{\mathbb{T}} % Torus or sometimes denoted for Topological space
\newcommand{\A}{\mathbb{A}} % Affine space
\newcommand{\0}{\mathbf{0}} % Zero vector

\newcommand{\tx}[1]{
	\text{#1}
	}
\newcommand{\mb}[1]{
	\mathbb{#1}
}

\newcommand{\seq}[3]{
% write sequence expression
% Params: #1 notation, #2,#3 lower and upper boundaries  
      \ensuremath{\{#1_n\}_{n=#2}^{#3}}
}


% tikz preset for relation
\newcommand{\tikzinitRelation}{
	\tikzset{
		circlenode/.style={
			circle,
			draw,
			thick,
			align=center 
		},
		edge/.style={
			->,          
			>={Latex},   
			thick        
		}
	}
	
}
% normal circle node with lable inside
\newcommand{\newnode}[3]{
	\node[circlenode] (#1) at (#2,#3) {#1};
}

% single arrow
\newcommand{\arrow}[2]{
	\draw[edge] (#1) -- (#2);
}

% arrow to self
\newcommand{\selfloop}[2]{
	\draw[edge, loop #2] (#1) to (#1);
}
%%%%%%%%%%%%%%%%%%%%%%%%%%%%%%%%%%%%%%%%%%%%%%%%%%%%%%%%%%%%%%%%%%%%%%%%%%%%%%%%%%%%%%%%
\newtheoremstyle{lemmastyle}
  {9pt}  
  {9pt} 
  {\slshape} 
  {}    
  {\bfseries} 
  {.}  
  {5pt} 
  {}   

\theoremstyle{lemmastyle}
\newtheorem{lemma}{Lemma}[chapter]



\newtheoremstyle{solutionStyle}
  {5pt}  
  {5pt} 
  {\upshape} 
  {\parindent} 
  {\bfseries}
  {:}    
  {3pt}  
  {}     


\theoremstyle{solutionStyle}  
\newtheorem*{solution}{Solution}

% Define a new theorem style called 'axiom_style'
\newtheoremstyle{axiom_style}% name of the style to be used
  {3pt}% Space above
  {3pt}% Space below
  {\normalfont}% Body font
  {}% Indent amount
  {\bfseries}% Theorem head font
  {.}% Punctuation after theorem head
  {.5em}% Space after theorem head
  {}% Theorem head spec (can be left empty, meaning 'normal')

% Apply the theorem style
\theoremstyle{axiom_style}

% Define the 'axiom' theorem with chapter numbering
\newtheorem{axiom}{Axiom}[chapter]
%\addbibresource{sample.bib} % Bibliography file

\definecolor{ocre}{RGB}{243, 102, 25} % Define the color used for highlighting throughout the book

\chapterimage{orange1.jpg} % Chapter heading image
\chapterspaceabove{6.5cm} % Default whitespace from the top of the page to the chapter title on chapter pages
\chapterspacebelow{6.75cm} % Default amount of vertical whitespace from the top margin to the start of the text on chapter pages

%----------------------------------------------------------------------------------------

\begin{document}

%----------------------------------------------------------------------------------------
%	TITLE PAGE
%----------------------------------------------------------------------------------------

\titlepage % Output the title page
	{\includegraphics[width=\paperwidth]{background.pdf}} % Code to output the background image, which should be the same dimensions as the paper to fill the page entirely; leave empty for no background image
	{ % Title(s) and author(s)
		\centering\sffamily % Font styling
		{\Huge\bfseries Miscellaneous Math Problems\par} % Book title
		\vspace{16pt} % Vertical whitespace
		{\LARGE A Temporary Math Resource} % Subtitle
		\vspace{24pt} % Vertical whitespace
		{\huge\bfseries \\ Eric Yang Xingyu\par} % Author name
	}

%----------------------------------------------------------------------------------------
%	COPYRIGHT PAGE
%----------------------------------------------------------------------------------------

\thispagestyle{empty} % Suppress headers and footers on this page

~\vfill % Push the text down to the bottom of the page

\noindent Copyright \copyright\ 2024 Eric Yang Xingyu\\ % Copyright notice

\noindent \textsc{Published by Publisher}\\ % Publisher
This \LaTeX \space template is from 
\noindent \textsc{\href{https://www.latextemplates.com/template/legrand-orange-book}{book-website.com}}\\ % URL

\noindent Licensed under the Creative Commons Attribution-NonCommercial 4.0 License (the ``License''). You may not use this file except in compliance with the License. You may obtain a copy of the License at \url{https://creativecommons.org/licenses/by-nc-sa/4.0}. Unless required by applicable law or agreed to in writing, software distributed under the License is distributed on an \textsc{``as is'' basis, without warranties or conditions of any kind}, either express or implied. See the License for the specific language governing permissions and limitations under the License.\\ % License information, replace this with your own license (if any)

\noindent \textit{First amendment, March 2024} % Printing/edition date

%----------------------------------------------------------------------------------------
%	TABLE OF CONTENTS
%----------------------------------------------------------------------------------------

\pagestyle{empty} % Disable headers and footers for the following pages

\tableofcontents % Output the table of contents

\listoffigures % Output the list of figures, comment or remove this command if not required

\listoftables % Output the list of tables, comment or remove this command if not required

\pagestyle{fancy} % Enable default headers and footers again

\cleardoublepage % Start the following content on a new page

%----------------------------------------------------------------------------------------
%	PART
%----------------------------------------------------------------------------------------

\part{Calculus and Real Analysis}
\chapter{Calculus and Mathematical Analysis}

\section{Function}

\subsection{Function Definition}

\begin{exercise}
Given \( f(x) = \frac{1-x}{1+x} \), find the values of \( f(0) \), \( f(-x) \), \( f(x+1) \), \( f(x) + 1 \), and \( f\left( \frac{1}{x} \right) \).
\end{exercise}
\begin{solution}
First, we calculate \( f(0) \):
\[
f(0) = \frac{1-0}{1+0} = \frac{1}{1} = 1
\]

Next, we find \( f(-x) \):
\[
f(-x) = \frac{1-(-x)}{1+(-x)} = \frac{1+x}{1-x}
\]

Then, we determine \( f(x+1) \):
\[
f(x+1) = \frac{1-(x+1)}{1+(x+1)} = \frac{1-x-1}{1+x+1} = \frac{-x}{x+2}
\]

Next, we compute \( f(x) + 1 \):
\[
f(x) + 1 = \frac{1-x}{1+x} + 1 = \frac{1-x + 1+x}{1+x} = \frac{2}{1+x}
\]

Finally, we find \( f\left( \frac{1}{x} \right) \):
\[
f\left( \frac{1}{x} \right) = \frac{1 - \frac{1}{x}}{1 + \frac{1}{x}} = \frac{\frac{x-1}{x}}{\frac{x+1}{x}} = \frac{x-1}{x+1}
\]
\end{solution}

\begin{exercise}
Find the domain of the following functions:
\begin{enumerate}
    \item \( y = (x-2) \sqrt{\frac{1+x}{1-x}} \)
    \item \( y = \sqrt{\cos x^2} \)
    \item \( y = \frac{\sqrt{x}}{\sin \pi x} \)
    \item \( y = \arcsin \frac{2x}{1+x} \)
    \item \( y = \arcsin (1-x) + \lg (\lg x) \)
\end{enumerate}
\end{exercise}

\begin{solution}
The domains of the given functions are as follows:
\begin{enumerate}
    \item For \( y = (x-2) \sqrt{\frac{1+x}{1-x}} \):
    \begin{itemize}
        \item The square root function requires the argument to be non-negative: \( \frac{1+x}{1-x} \geq 0 \).
        \item The denominator \( 1-x \) must be positive (to avoid division by zero and ensure the expression inside the square root is defined):
        \[
        1-x > 0 \implies x < 1
        \]
        \item Combining these conditions, we have:
        \[
        -1 \leq x < 1
        \]
    \end{itemize}
    
    \item For \( y = \sqrt{\cos x^2} \):
    \begin{itemize}
        \item The argument of the square root must be non-negative: \( \cos x^2 \geq 0 \).
        \item Cosine function is non-negative when \( x^2 \) lies within intervals where cosine is non-negative:
        \[
        \cos x^2 \geq 0 \implies x^2 \in \left[2k\pi - \frac{\pi}{2}, 2k\pi + \frac{\pi}{2}\right] \quad \text{for integer } k.
        \]
        \item Solving for \( x \):
        \[
        x \in \left[-\sqrt{2k\pi + \frac{\pi}{2}}, \sqrt{2k\pi + \frac{\pi}{2}}\right]
        \]
    \end{itemize}

    \item For \( y = \frac{\sqrt{x}}{\sin \pi x} \):
    \begin{itemize}
        \item The numerator \( \sqrt{x} \) requires \( x \geq 0 \).
        \item The denominator \( \sin \pi x \) must be non-zero:
        \[
        \sin \pi x \neq 0 \implies x \notin \mathbb{Z}
        \]
        \item Combining these conditions, we have:
        \[
        x \geq 0 \quad \text{and} \quad x \notin \mathbb{Z}
        \]
    \end{itemize}

    \item For \( y = \arcsin \frac{2x}{1+x} \):
    \begin{itemize}
        \item The argument of the arcsine function must lie within \([-1, 1]\):
        \[
        -1 \leq \frac{2x}{1+x} \leq 1
        \]
        \item Solving the inequalities:
        \[
        \frac{2x}{1+x} \leq 1 \implies 2x \leq 1 + x \implies x \leq 1
        \]
        \[
        \frac{2x}{1+x} \geq -1 \implies 2x \geq -1 - x \implies x \geq -\frac{1}{3}
        \]
        \item Combining these conditions, we have:
        \[
        -\frac{1}{3} \leq x \leq 1
        \]
    \end{itemize}

    \item For \( y = \arcsin (1-x) + \lg (\lg x) \):
    \begin{itemize}
        \item The argument of the arcsine function must lie within \([-1, 1]\):
        \[
        -1 \leq 1-x \leq 1 \implies 0 \leq x \leq 2
        \]
        \item The argument of the logarithmic function must be positive:
        \[
        \lg x > 0 \implies x > 1
        \]
        \item Combining these conditions, we have:
        \[
        1 < x \leq 2
        \]
    \end{itemize}
\end{enumerate}
\end{solution}

\begin{exercise}
Find the domain and range of the following functions:
\begin{enumerate}
    \item \( y = \sqrt{2 + x - x^2} \)
    \item \( y = \arccos \frac{2x}{1+x^2} \)
    \item \( y = \sqrt{x - x^2} \)
\end{enumerate}
\end{exercise}
\begin{remark}
    Given a quadratic function \( f(x) = ax^2 + bx + c \):

\begin{enumerate}
    \item Determine the direction of the parabola:
    \begin{itemize}
        \item If \( a > 0 \), the parabola opens upwards and \( f(x) \) has a minimum value.
        \item If \( a < 0 \), the parabola opens downwards and \( f(x) \) has a maximum value.
    \end{itemize}

    \item Find the vertex:
    \begin{itemize}
        \item The \( x \)-coordinate of the vertex is \( h = -\frac{b}{2a} \).
        \item The \( y \)-coordinate of the vertex (the extremum value) is \( k = f(h) = c - \frac{b^2}{4a} \).
    \end{itemize}
    
    \item Conclusion:
    \begin{itemize}
        \item If \( a > 0 \), the minimum value of \( f(x) \) is \( k = c - \frac{b^2}{4a} \).
        \item If \( a < 0 \), the maximum value of \( f(x) \) is \( k = c - \frac{b^2}{4a} \).
    \end{itemize}
\end{enumerate}
\end{remark}
\begin{solution}
The domains and ranges of the given functions are as follows:
\begin{enumerate}
    \item For \( y = \sqrt{2 + x - x^2} \):
    \begin{itemize}
        \item The argument of the square root must be non-negative:
        \[
        2 + x - x^2 \geq 0
        \]
        \item This is a quadratic inequality, solving for \( x \):
        \[
        x^2 - x - 2 \leq 0 \implies (x-2)(x+1) \leq 0
        \]
        \item The solutions are:
        \[
        -1 \leq x \leq 2
        \]
        \item The range of \( y \) is determined by the maximum value of \( 2 + x - x^2 \) within the domain. The maximum value occurs at \( x = \frac{1}{2}(2+1) = 1.5 \):
        \[
        y_{\max} = \sqrt{2 + 1.5 - 1.5^2} = \sqrt{\frac{9}{4}} = \frac{3}{2}
        \]
        \item Therefore:
        \[
        0 \leq y \leq \frac{3}{2}
        \]
    \end{itemize}

    \item For \( y = \arccos \frac{2x}{1+x^2} \):
    \begin{itemize}
        \item The argument of the arcsine function must lie within \([-1, 1]\):
        \[
        -1 \leq \frac{2x}{1+x^2} \leq 1
        \]
        \item This inequality holds for all \( x \in \mathbb{R} \).
        \item The range of the \( \arccos \) function is:
        \[
        0 \leq y \leq \pi
        \]
    \end{itemize}

    \item For \( y = \sqrt{x - x^2} \):
    \begin{itemize}
        \item The argument of the square root must be non-negative:
        \[
        x - x^2 \geq 0
        \]
        \item This is a quadratic inequality, solving for \( x \):
        \[
        x(1 - x) \geq 0
        \]
        \item The solutions are:
        \[
        0 \leq x \leq 1
        \]
        \item The range of \( y \) is determined by the maximum value of \( x - x^2 \) within the domain. The maximum value occurs at \( x = \frac{1}{2} \):
        \[
        y_{\max} = \sqrt{\frac{1}{4}} = \frac{1}{2}
        \]
        \item Therefore:
        \[
        0 \leq y \leq \frac{1}{2}
        \]
    \end{itemize}
\end{enumerate}
\end{solution}

\begin{exercise}
\begin{enumerate}
    \item Given \( f\left( x + \frac{1}{x} \right) = x^2 + \frac{1}{x^2} \), find \( f(x) \).
    \item Given \( f\left( \frac{1}{x} \right) = x + \sqrt{1 + x^2} \) for \( x > 0 \), find \( f(x) \).
\end{enumerate}
\end{exercise}
\begin{solution}
\begin{enumerate}
    \item For the function \( f\left( x + \frac{1}{x} \right) = x^2 + \frac{1}{x^2} \), we need to find \( f(x) \).
    \begin{itemize}
        \item Let \( t = x + \frac{1}{x} \). Then we have:
        \[
        f(t) = x^2 + \frac{1}{x^2}
        \]
        \item We express \( x^2 + \frac{1}{x^2} \) in terms of \( t \):
        \[
        t^2 = \left( x + \frac{1}{x} \right)^2 = x^2 + 2 + \frac{1}{x^2}
        \]
        \item Thus, we get:
        \[
        x^2 + \frac{1}{x^2} = t^2 - 2
        \]
        \item Therefore:
        \[
        f(t) = t^2 - 2
        \]
        \item Hence, the function is:
        \[
        f(x) = x^2 - 2
        \]
    \end{itemize}

    \item For the function \( f\left( \frac{1}{x} \right) = x + \sqrt{1 + x^2} \) where \( x > 0 \), we need to find \( f(x) \).
    \begin{itemize}
        \item Let \( t = \frac{1}{x} \). Then:
        \[
        f(t) = x + \sqrt{1 + x^2}
        \]
        \item Substituting \( x = \frac{1}{t} \), we get:
        \[
        f(t) = \frac{1}{t} + \sqrt{1 + \left( \frac{1}{t} \right)^2} = \frac{1}{t} + \sqrt{\frac{t^2 + 1}{t^2}} = \frac{1}{t} + \frac{\sqrt{t^2 + 1}}{t} = \frac{1 + \sqrt{t^2 + 1}}{t}
        \]
        \item Thus:
        \[
        f(x) = \frac{1 + \sqrt{x^2 + 1}}{x}
        \]
    \end{itemize}
\end{enumerate}
\end{solution}



\begin{exercise}
In the open interval \(0 < x < 1\) on the \(O x\) axis, there is a uniformly distributed mass of \(2 \mathrm{~kg}\). At the points \(x = 2\) and \(x = 3\), there are concentrated masses of \(1 \mathrm{~kg}\) each. Let \(m(x)\) be the total mass in the interval \((-\infty, x)\). Find the analytical expression for the function \(m(x)\) in terms of a piecewise function and plot this function.
\end{exercise}
\begin{solution}
First, determine the expression for \( m(x) \) in different intervals:

1. For \( x \leq 0 \): There is no mass in the interval \((-\infty, x)\):
   \[
   m(x) = 0
   \]

2. For \( 0 < x \leq 1 \): The mass is uniformly distributed in the interval \( 0 < x < 1 \) with a total mass of \(2 \mathrm{~kg}\):
   \[
   m(x) = 2x
   \]

3. For \( 1 < x \leq 2 \): The entire \(2 \mathrm{~kg}\) mass in the interval \( 0 < x < 1 \) is considered:
   \[
   m(x) = 2 \mathrm{~kg}
   \]

4. For \( 2 < x \leq 3 \): In addition to the \(2 \mathrm{~kg}\) mass, there is a concentrated mass of \(1 \mathrm{~kg}\) at \( x = 2 \):
   \[
   m(x) = 3 \mathrm{~kg}
   \]

5. For \( x > 3 \): There are concentrated masses of \(1 \mathrm{~kg}\) each at \( x = 2 \) and \( x = 3 \):
   \[
   m(x) = 4 \mathrm{~kg}
   \]

The piecewise function \( m(x) \) can be written as:
\[
m(x) =
\begin{cases}
0, & x \leq 0 \\
2x, & 0 < x \leq 1 \\
2, & 1 < x \leq 2 \\
3, & 2 < x \leq 3 \\
4, & x > 3
\end{cases}
\]

\begin{center}
\begin{tikzpicture}
    \begin{scope}[xshift=0cm]
        \draw[->] (-1,0) -- (5,0) node[right] {$x$};
        \draw[->] (0,-1) -- (0,5) node[above] {$m(x)$};
        
        % Drawing the mass function
        \draw[thick, domain=-1:0] plot (\x, {0}) node[right] {};
        \draw[thick, domain=0:1] plot (\x, {2*\x}) node[right] {};
        \draw[thick] (1,2) -- (2,2) node[right] {};
        %\draw[thick] (2,2) -- (2,3) node[right] {};
        \draw[thick] (2,3) -- (3,3) node[right] {};
        %\draw[thick] (3,3) -- (3,4) node[right] {};
        \draw[thick, domain=3:5] plot (\x, {4}) node[right] {};
        
        % Adding the points
        \filldraw[black] (0,0) circle (2pt);
        \filldraw[black] (1,2) circle (2pt);
        \draw[fill=white] (2,2) circle (2pt);
        \filldraw[black] (2,3) circle (2pt);
        \draw[fill=white] (3,3) circle (2pt);
        \filldraw[black] (3,4) circle (2pt);

        % Adding labels
        \node at (1,-0.3) {$1$};
        \node at (2,-0.3) {$2$};
        \node at (3,-0.3) {$3$};
        \node at (-0.3,2) {$2$};
        \node at (-0.3,3) {$3$};
        \node at (-0.3,4) {$4$};
    \end{scope}
    
   
\end{tikzpicture}
\end{center}

\end{solution}

\begin{exercise}
Let \( f(x) \) be defined as follows:
\[
f(x) =
\begin{array}{ll}
1, & \text{if } |x| \leq 1 \\
0, & \text{if } |x| > 1
\end{array}
\]
Find \( f(f(x)) \).
\end{exercise}
\begin{solution}
We need to find the expression for \( f(f(x)) \).

First, consider the given function \( f(x) \):
\[
f(x) =
\begin{array}{ll}
1, & \text{if } |x| \leq 1 \\
0, & \text{if } |x| > 1
\end{array}
\]

We will analyze \( f(f(x)) \) by considering the two possible cases for \( f(x) \):

1. When \( |x| \leq 1 \):
   \[
   f(x) = 1
   \]
   Substituting \( f(x) = 1 \) into \( f \):
   \[
   f(f(x)) = f(1)
   \]
   Since \( 1 \) satisfies \( |1| \leq 1 \):
   \[
   f(1) = 1
   \]
   Therefore, when \( |x| \leq 1 \):
   \[
   f(f(x)) = 1
   \]

2. When \( |x| > 1 \):
   \[
   f(x) = 0
   \]
   Substituting \( f(x) = 0 \) into \( f \):
   \[
   f(f(x)) = f(0)
   \]
   Since \( 0 \) satisfies \( |0| \leq 1 \):
   \[
   f(0) = 1
   \]
   Therefore, when \( |x| > 1 \):
   \[
   f(f(x)) = 1
   \]

In both cases, we have \( f(f(x)) = 1 \). Hence, the function \( f(f(x)) \) is:
\[
f(f(x)) = 1
\]

\end{solution}

\begin{exercise}
Let \( f_n(x) = \underbrace{f(f(\cdots f(x)))}_{n \text{ times}} \). Given \( f(x) = \frac{x}{\sqrt{1+x^2}} \), find \( f_n(x) \).
\end{exercise}

\begin{solution}
First, we calculate \( f(f(x)) \):
\[
f(f(x)) = f\left( \frac{x}{\sqrt{1 + x^2}} \right)
\]
Substituting the definition of \( f \):
\[
f\left( \frac{x}{\sqrt{1 + x^2}} \right) = \frac{\frac{x}{\sqrt{1 + x^2}}}{\sqrt{1 + \left( \frac{x}{\sqrt{1 + x^2}} \right)^2}} = \frac{\frac{x}{\sqrt{1 + x^2}}}{\sqrt{\frac{1 + 2x^2}{1 + x^2}}} = \frac{x}{\sqrt{1 + 2x^2}}
\]

Thus,
\[
f(f(x)) = \frac{x}{\sqrt{1 + 2x^2}}
\]

Next, we calculate \( f(f(f(x))) \):
\[
f(f(f(x))) = f\left( \frac{x}{\sqrt{1 + 2x^2}} \right) = \frac{x}{\sqrt{1 + 3x^2}}
\]

From the pattern observed, we can generalize:
\[
f_n(x) = \frac{x}{\sqrt{1 + nx^2}}
\]

We will use mathematical induction to prove:
\[
f_n(x) = \frac{x}{\sqrt{1 + nx^2}}
\]

\textbf{Base Case:} For \( n = 1 \):
\[
f_1(x) = f(x) = \frac{x}{\sqrt{1 + x^2}}
\]
Clearly,
\[
f_1(x) = \frac{x}{\sqrt{1 + 1 \cdot x^2}} = \frac{x}{\sqrt{1 + x^2}}
\]
Thus, the proposition holds for \( n = 1 \).

\textbf{Induction Hypothesis:} Assume the proposition holds for \( n = k \), i.e.,
\[
f_k(x) = \frac{x}{\sqrt{1 + kx^2}}
\]

\textbf{Inductive Step:} We need to prove that the proposition holds for \( n = k+1 \), i.e.,
\[
f_{k+1}(x) = \frac{x}{\sqrt{1 + (k+1)x^2}}
\]

By definition,
\[
f_{k+1}(x) = f(f_k(x))
\]
Substitute the induction hypothesis:
\[
f_{k+1}(x) = f\left( \frac{x}{\sqrt{1 + kx^2}} \right)
\]
Using the definition of \( f(x) \):
\[
f\left( \frac{x}{\sqrt{1 + kx^2}} \right) = \frac{\frac{x}{\sqrt{1 + kx^2}}}{\sqrt{1 + \left( \frac{x}{\sqrt{1 + kx^2}} \right)^2}}
\]
Simplify the expression inside the square root:
\[
\left( \frac{x}{\sqrt{1 + kx^2}} \right)^2 = \frac{x^2}{1 + kx^2}
\]
Thus,
\[
1 + \left( \frac{x}{\sqrt{1 + kx^2}} \right)^2 = 1 + \frac{x^2}{1 + kx^2} = \frac{1 + kx^2 + x^2}{1 + kx^2} = \frac{1 + (k+1)x^2}{1 + kx^2}
\]
Therefore,
\[
f\left( \frac{x}{\sqrt{1 + kx^2}} \right) = \frac{\frac{x}{\sqrt{1 + kx^2}}}{\sqrt{\frac{1 + (k+1)x^2}{1 + kx^2}}} = \frac{x}{\sqrt{1 + (k+1)x^2}}
\]

So,
\[
f_{k+1}(x) = \frac{x}{\sqrt{1 + (k+1)x^2}}
\]

This shows that if the proposition holds for \( n = k \), then it also holds for \( n = k+1 \).

By mathematical induction, the proposition holds for all natural numbers \( n \).

Thus, for the given function \( f(x) = \frac{x}{\sqrt{1 + x^2}} \), the \( n \)-th iteration \( f_n(x) \) is:
\[
f_n(x) = \frac{x}{\sqrt{1 + nx^2}}
\]
\end{solution}

\subsection{Function Parity}
\subsection{Function Periodicity}
\subsection{Function Monotonicity}
\subsection{Function Boundedness}
\begin{definition}[Bounded Above and Below]
  Let \( f \) be a function defined on a set \( D \). If there exists a constant \( M \in \mathbb{R} \) such that for every \( x \in D \), the inequality 
  \[
  f(x) \leq M \quad (\text{or } f(x) \geq N)
  \]
  holds, then \( f \) is called bounded above (or bounded below) on \( D \) by \( M \) (or \( N \)). Formally, if \( M \) (or \( N \)) is an upper (or lower) bound of \( f \), any number less than (or greater than) \( M \) (or \( N \)) is also an upper (or lower) bound of \( f \) on \( D \). That is,
  \[
  \exists M \in \mathbb{R}, \forall x \in D, f(x) \leq M \quad (\text{or } \exists N \in \mathbb{R}, \forall x \in D, f(x) \geq N).
  \]
\end{definition}

\begin{definition}[Bounded Function]
  Let \( f \) be a function defined on a set \( D \). If there exist constants \( N, M \in \mathbb{R} \) such that \( N \leq M \) and for every \( x \in D \), the inequality 
  \[
  N \leq f(x) \leq M
  \]
  holds, then \( f \) is called bounded on \( D \). We often express this by saying: A function \( f \) defined on \( D \) is called bounded if there exists a constant \( M > 0 \) such that for every \( x \in D \), the inequality 
  \[
  |f(x)| \leq M
  \]
  holds. Formally,
  \[
  \exists M \in \mathbb{R}, \forall x \in D, |f(x)| \leq M.
  \]
  In this case, if \( f \) is bounded on \( D \), its graph will lie completely between the lines \( y = M \) and \( y = -M \).
\end{definition}

\begin{definition}[Unbounded Function]
  Let \( f \) be a function defined on a set \( D \). If for every positive constant \( M \in \mathbb{R} \), there exists \( x_\infty \in D \) such that the inequality 
  \[
  |f(x_\infty)| > M
  \]
  holds, then \( f \) is called unbounded on \( D \). Formally,
  \[
  \forall M > 0, \exists x_\infty \in D, |f(x_\infty)| > M.
  \]
\end{definition}

\begin{example}
    Prove that $f: \R \to \R, f(x)=\frac{1}{\sqrt{x}}$ is unbounded on $(0,1]$.
\end{example}
\begin{proof}
We define $M \in (0,1]$, and $x_0 = \frac{1}{(M+1)^2}$. 
$$
\left|f\left(x_0\right)\right|=\left|\frac{1}{\sqrt{x_0}}\right|=M+1>M.
$$
\end{proof}

\begin{exercise}
    Show that the sup and inf of a set are uniquely determined whenever they exist.
\end{exercise}
\begin{solution}
We will prove that the supremum (sup) of a set is uniquely determined whenever it exists. The proof for the infimum (inf) follows similarly.

Given a nonempty set $S \subseteq \mathbb{R}$, let's assume that $S$ has two suprema: $\sup S = a$ and $\sup S = b$. We will show that $a = b$.

\begin{proof}
Suppose, for the sake of contradiction, that $a \neq b$. Without loss of generality, we can assume $a > b$. (The case $a < b$ can be proved similarly.)

Let $\varepsilon = \frac{a-b}{2}$. Note that $\varepsilon > 0$ since $a > b$.

By the definition of supremum, for any $\varepsilon > 0$, there exists an $x \in S$ such that
\[
a - \varepsilon < x \leq a
\]

Substituting our chosen $\varepsilon$, we have:
\[
a - \frac{a-b}{2} < x \leq a
\]

Simplifying the left side:
\[
\frac{a+b}{2} < x \leq a
\]

Now, observe that:
\[
b < \frac{a+b}{2} = a - \varepsilon < x < a
\]

This implies that
\[
b < x
\]

However, this contradicts the assumption that $b = \sup S$, because we have found an element $x \in S$ that is greater than $b$.

Therefore, our initial assumption that $a \neq b$ must be false. We conclude that $a = b$.

Thus, we have shown that if the supremum of a set exists, it must be unique. The proof for the uniqueness of the infimum follows a similar structure, considering the case where we assume two different infima and deriving a contradiction.

\end{proof}
\end{solution}
\begin{exercise}
Determine the boundedness of the following functions:
\begin{enumerate}
    \item \( y = |\sin x| \mathrm{e}^{\cos x} \)
    \item \( y = \frac{x}{1 + x^2} \)
    \item \( y = \sin \frac{1}{x} \)
    \item \( y = \mathrm{e}^{\frac{1}{x}} \)
\end{enumerate}
\end{exercise}
\begin{solution}
\begin{enumerate}
    \item \( y = |\sin x| \mathrm{e}^{\cos x} \)
    
    Since \( |\sin x| \leq 1 \) and \( |\cos x| \leq 1 \), we get:
    \[
    |\sin x| \mathrm{e}^{\cos x} \leq \mathrm{e}^{|\cos x|} \leq \mathrm{e}
    \]
    Therefore, the function \( y \) is bounded with:
    \[
    0 \leq y \leq \mathrm{e}
    \]
    
    \item \( y = \frac{x}{1 + x^2} \)
    
    For \( x \neq 0 \):
    \[
    \left| \frac{x}{1 + x^2} \right| \leq \frac{|x|}{1 + x^2} \leq \frac{1}{2}
    \]
    When \( x = 0 \), the above inequality still holds. Thus, for all \( x \in \mathbb{R} \):
    \[
    \left| \frac{x}{1 + x^2} \right| \leq \frac{1}{2}
    \]
    Therefore, the function \( y \) is bounded with:
    \[
    -\frac{1}{2} \leq y \leq \frac{1}{2}
    \]
    
    \item \( y = \sin \frac{1}{x} \)
    
    Since \( \left| \sin \frac{1}{x} \right| \leq 1 \), the function \( y \) is bounded with:
    \[
    -1 \leq y \leq 1
    \]
    
    \item \( y = \mathrm{e}^{\frac{1}{x}} \)
    
    For any \( M > 0 \), let \( 0 < x_0 < \frac{1}{\ln M} \). Then:
    \[
    \frac{1}{x_0} > \ln M
    \]
    Thus:
    \[
    \mathrm{e}^{\frac{1}{x_0}} > \mathrm{e}^{\ln M} = M
    \]
    which means:
    \[
    \left| \mathrm{e}^{\frac{1}{x_0}} \right| > M
    \]
    Therefore, the function \( y \) is unbounded.
\end{enumerate}
\end{solution}

\begin{exercise}
Determine whether the function \( f(x) = \frac{\sin(x)}{x} \) is bounded on the interval \( (0, \infty) \). Note that you don't need to show the upper and lower boundaries, since we cannot get them yet. Show this by showing that every possible value of $f(x)$ in the domain is bounded by some other function.
\end{exercise}
\begin{solution}
We first consider the absolute value of \( f(x) \).
\[
|f(x)| = \left| \frac{\sin(x)}{x} \right| = \frac{|\sin(x)|}{|x|} = \frac{|\sin(x)|}{x}
\]
We know that \( |\sin(x)| \in [0, 1] \). So we have
\[
0 \leq \frac{|\sin(x)|}{x} \leq \frac{1}{x}
\]

So, \[\exists M\in \R, \text{where } M = \frac{1}{x}, \forall x \in (0, \infty), |f(x)| \leq M.\]
That is to say, the function \( f(x) = \frac{\sin(x)}{x} \) is bounded by some value $\frac{1}{x}$ on \( (0, \infty) \). However, we cannot calculate the exact bounded value for now, and we will discuss this in the future.
\end{solution}
%

\section{Limit of Sequence}
\begin{definition}[$\varepsilon-N$ Definition of Limits on Sequence]
Let $\{a_n\}$ be a sequence and let $a$ be a constant. If for any given positive number $\varepsilon$, there exists a positive integer $N$ such that whenever $n > N$, we have $\left|a_n - a\right| < \varepsilon$, then $a$ is called the limit of the sequence $\{a_n\}$. In other words, the sequence $\{a_n\}$ converges to $a$, denoted as $\lim_{n \to \infty} a_n = a$ (read as "the limit of $a_n$ as $n$ approaches infinity is $a$") or $a_n \to a$ as $n \to \infty$ (read as "$a_n$ approaches $a$ as $n$ approaches infinity").

This definition is known as the $\varepsilon$-$N$ definition of the limit of a sequence.

The $\varepsilon$-$N$ definition can be stated as:

If $\forall \varepsilon > 0$, $\exists N\in \Z^+$ such that whenever $n > N$, we have $\left|a_n - a\right| < \varepsilon$, then $\lim_{n \to \infty} a_n = a$.
\end{definition}

\begin{theorem}[Uniqueness of Limits]
If a sequence $\{a_n\}$ has a limit, then that limit is unique. That is, if $\lim_{n \to \infty} a_n = L$ and $\lim_{n \to \infty} a_n = M$, then $L = M$.
\end{theorem}
\begin{proof}
Assume that $\{a_n\}$ is a sequence such that $\lim_{n \to \infty} a_n = L$ and $\lim_{n \to \infty} a_n = M$, where $L$ and $M$ are two possibly distinct limits. We need to show that $L = M$.

By the definition of limits, for any $\varepsilon > 0$, there exists a positive integer $N_1$ such that for all $n > N_1$, $|a_n - L| < \frac{\varepsilon}{2}$. Similarly, there exists a positive integer $N_2$ such that for all $n > N_2$, $|a_n - M| < \frac{\varepsilon}{2}$.

Let $N = \max\{N_1, N_2\}$. For all $n > N$, we have both $|a_n - L| < \frac{\varepsilon}{2}$ and $|a_n - M| < \frac{\varepsilon}{2}$. Therefore,
\[
|L - M| = |(L - a_n) + (a_n - M)| \leq |L - a_n| + |a_n - M| < \frac{\varepsilon}{2} + \frac{\varepsilon}{2} = \varepsilon.
\]

Since $\varepsilon$ is arbitrary, we can make $\varepsilon$ as small as we like. The only way for $|L - M|$ to be less than any positive number $\varepsilon$ is if $|L - M| = 0$. Therefore, $L = M$.

This completes the proof that the limit of a sequence, if it exists, is unique.
\end{proof}
%
\begin{theorem}[Boundedness of Convergent Sequences]
If a sequence $\{a_n\}$ converges, then $\{a_n\}$ is bounded. That is, there exists a constant $M$ such that for all positive integers $n$, we have $\left|a_n\right| \leq M$.
\end{theorem}
\begin{proof}
Let $\lim_{n \to \infty} a_n = a$. By the definition of limits, for $\varepsilon = 1$, there exists a positive integer $N$ such that for all $n > N$, we have
\[
\left|a_n - a\right| < 1.
\]
Thus, for $n > N$, we have $\left|a_n\right| \leq \left|a\right| + 1$. Define
\[
M = \max \left\{\left|a_1\right|, \left|a_2\right|, \ldots, \left|a_N\right|, 1 + \left|a\right|\right\}.
\]
Therefore, for all positive integers $n$, we have $\left|a_n\right| \leq M$.

This shows that $\{a_n\}$ is bounded.
\end{proof}
\begin{remark}
Boundedness is a necessary condition for the convergence of a sequence but not a sufficient condition. For example, the sequence $\{(-1)^n\}$ is bounded but does not converge. The contrapositive of this property is true, namely:

\begin{corollary}
If a sequence $\{a_n\}$ is unbounded, then it diverges.
\end{corollary}
\end{remark}
%
\begin{theorem}[Comparison Property of Limits]
If $\lim_{n \to \infty} a_n = a$ and $\lim_{n \to \infty} b_n = b$ with $a < b$, then there exists a positive integer $N$ such that for all $n > N$ (i.e., for sufficiently large $n$), we have $a_n < b_n$.
\end{theorem}
\begin{proof}
Since $\lim_{n \to \infty} a_n = a$ and $\lim_{n \to \infty} b_n = b$ with $a < b$, let $\varepsilon = \frac{b - a}{2} > 0$. There exist positive integers $N_1$ and $N_2$ such that
\[
\text{for } n > N_1, \left|a_n - a\right| < \frac{b - a}{2} \text{, which implies } \frac{3a - b}{2} < a_n < \frac{a + b}{2},
\]
and
\[
\text{for } n > N_2, \left|b_n - b\right| < \frac{b - a}{2} \text{, which implies } \frac{a + b}{2} < b_n < \frac{3b - a}{2}.
\]

Let $N = \max \left\{N_1, N_2\right\}$. For all $n > N$, we have
\[
a_n < \frac{a + b}{2} \text{ and } \frac{a + b}{2} < b_n \text{, hence } a_n < b_n.
\]
\end{proof}

\begin{corollary}[Positivity(Negativity) Preservation]
If $\lim_{n \to \infty} a_n = a > 0$ (or $\lim_{n \to \infty} a_n = a < 0$), then for any constant $\eta$ satisfying $0 < \eta < a$ (or $a < \eta < 0$), there exists a positive integer $N$ such that for all $n > N$, we have
\[
a_n > \eta > 0 \quad \left(a_n < \eta < 0\right).
\]
\end{corollary}
\begin{proof}
For $0 < \eta < a$, let $b_n = \eta$ for all $n$. Then $\lim_{n \to \infty} b_n = \eta$. Given $\lim_{n \to \infty} a_n = a > 0$ and $0 < \eta < a$, by the definition of limits, there exists a positive integer $N_1$ such that for all $n > N_1$, we have
\[
|a_n - a| < a - \eta.
\]
This implies that $a_n > a - (a - \eta) = \eta$ for all $n > N_1$.

Similarly, for $a < \eta < 0$, let $b_n = \eta$ for all $n$. Then $\lim_{n \to \infty} b_n = \eta$. Given $\lim_{n \to \infty} a_n = a < 0$ and $a < \eta < 0$, by the definition of limits, there exists a positive integer $N_2$ such that for all $n > N_2$, we have
\[
|a_n - a| < \eta - a.
\]
This implies that $a_n < a + (\eta - a) = \eta$ for all $n > N_2$.

Let $N = \max\{N_1, N_2\}$. For all $n > N$, we have $a_n > \eta > 0$ (or $a_n < \eta < 0$).
\end{proof}


% operational properties
\begin{theorem}[Limit of Sum and Difference]
If $\lim_{n \to \infty} a_n = L$ and $\lim_{n \to \infty} b_n = M$, then
\[
\lim_{n \to \infty} (a_n \pm b_n) = L \pm M.
\]
\end{theorem}
\begin{proof}
Let $\varepsilon > 0$. Since $\lim_{n \to \infty} a_n = L$ and $\lim_{n \to \infty} b_n = M$, there exist positive integers $N_1$ and $N_2$ such that for all $n > N_1$, $|a_n - L| < \frac{\varepsilon}{2}$, and for all $n > N_2$, $|b_n - M| < \frac{\varepsilon}{2}$. Let $N = \max\{N_1, N_2\}$. For all $n > N$, we have
\[
|a_n + b_n - (L + M)| \leq |a_n - L| + |b_n - M| < \frac{\varepsilon}{2} + \frac{\varepsilon}{2} = \varepsilon.
\]
Thus, $\lim_{n \to \infty} (a_n + b_n) = L + M$.

Similarly, for the difference,
\[
|a_n - b_n - (L - M)| \leq |a_n - L| + |b_n - M| < \frac{\varepsilon}{2} + \frac{\varepsilon}{2} = \varepsilon.
\]
Thus, $\lim_{n \to \infty} (a_n - b_n) = L - M$.
\end{proof}
%
\begin{theorem}[Limit of Product]
If $\lim_{n \to \infty} a_n = L$ and $\lim_{n \to \infty} b_n = M$, then
\[
\lim_{n \to \infty} (a_n b_n) = LM.
\]
\end{theorem}
\begin{proof}
Let $\varepsilon > 0$. Since $\lim_{n \to \infty} a_n = L$ and $\lim_{n \to \infty} b_n = M$, there exist positive integers $N_1$ and $N_2$ such that for all $n > N_1$, $|a_n - L| < \frac{\varepsilon}{2(|M| + 1)}$, and for all $n > N_2$, $|b_n - M| < \frac{\varepsilon}{2(|L| + 1)}$. Let $N = \max\{N_1, N_2\}$. For all $n > N$, we have
\[
|a_n b_n - LM| = |a_n b_n - a_n M + a_n M - LM| \leq |a_n||b_n - M| + |M||a_n - L|.
\]
Since $|a_n| \leq |L| + 1$ for sufficiently large $n$ and $|b_n| \leq |M| + 1$ for sufficiently large $n$, we get
\[
|a_n b_n - LM| < (|L| + 1)\frac{\varepsilon}{2(|L| + 1)} + (|M| + 1)\frac{\varepsilon}{2(|M| + 1)} = \frac{\varepsilon}{2} + \frac{\varepsilon}{2} = \varepsilon.
\]
Thus, $\lim_{n \to \infty} (a_n b_n) = LM$.
\end{proof}
%
\begin{theorem}[Limit of Quotient]
If $\lim_{n \to \infty} a_n = L$ and $\lim_{n \to \infty} b_n = M$ with $M \neq 0$, then
\[
\lim_{n \to \infty} \left(\frac{a_n}{b_n}\right) = \frac{L}{M}.
\]
\end{theorem}
\begin{proof}
Let $\varepsilon > 0$. Since $\lim_{n \to \infty} a_n = L$ and $\lim_{n \to \infty} b_n = M$, there exist positive integers $N_1$ and $N_2$ such that for all $n > N_1$, $|a_n - L| < \frac{\varepsilon |M|}{2}$, and for all $n > N_2$, $|b_n - M| < \frac{|M|}{2}$. Let $N = \max\{N_1, N_2\}$. For all $n > N$, we have $|b_n| \geq \frac{|M|}{2}$ and
\[
\left|\frac{a_n}{b_n} - \frac{L}{M}\right| = \left|\frac{a_n M - L b_n}{b_n M}\right| = \left|\frac{a_n M - L b_n}{b_n M}\right| \leq \left|\frac{a_n - L}{b_n}\right| + \left|\frac{L(b_n - M)}{b_n M}\right|.
\]
For sufficiently large $n$, we have
\[
\left|\frac{a_n - L}{b_n}\right| < \frac{\varepsilon |M|}{2 \cdot \frac{|M|}{2}} = \frac{\varepsilon}{2} \quad \text{and} \quad \left|\frac{L (b_n - M)}{b_n M}\right| < \frac{|L| \cdot \frac{|M|}{2}}{\frac{|M|}{2} \cdot |M|} = \frac{\varepsilon}{2}.
\]
Thus,
\[
\left|\frac{a_n}{b_n} - \frac{L}{M}\right| < \frac{\varepsilon}{2} + \frac{\varepsilon}{2} = \varepsilon.
\]
Therefore, $\lim_{n \to \infty} \left(\frac{a_n}{b_n}\right) = \frac{L}{M}$.
\end{proof}
%
\begin{corollary}[Limits of Polynomial Division]
For the limits of polynomial sequences like
\[
\lim_{n \to \infty} \frac{a_0 n^m + a_1 n^{m-1} + \cdots + a_{m-1} n + a_m}{b_0 n^k + b_1 n^{k-1} + \cdots + b_{k-1} n + b_k}.
\]
where \(m \leq k\), \(a_0 \neq 0\), and \(b_0 \neq 0\).
\[
\lim_{n \to \infty} \frac{a_0 n^m + a_1 n^{m-1} + \cdots + a_{m-1} n + a_m}{b_0 n^k + b_1 n^{k-1} + \cdots + b_{k-1} n + b_k} = 
\begin{cases} 
\frac{a_0}{b_0}, & \text{if } m = k, \\
0, & \text{if } m < k, \\
\infty, & \text{if } m > k.
\end{cases}
\]
\end{corollary}
\begin{proof}
We need to find the limit
\[
\lim_{n \to \infty} \frac{a_0 n^m + a_1 n^{m-1} + \cdots + a_{m-1} n + a_m}{b_0 n^k + b_1 n^{k-1} + \cdots + b_{k-1} n + b_k}.
\]

First, factor \(n^m\) from the numerator and \(n^k\) from the denominator:
\[
\lim_{n \to \infty} \frac{n^m \left(a_0 + a_1 \frac{1}{n} + \cdots + a_{m-1} \frac{1}{n^{m-1}} + a_m \frac{1}{n^m}\right)}{n^k \left(b_0 + b_1 \frac{1}{n} + \cdots + b_{k-1} \frac{1}{n^{k-1}} + b_k \frac{1}{n^k}\right)}.
\]

Simplify the expression:
\[
\lim_{n \to \infty} \frac{n^{m-k} \left(a_0 + a_1 \frac{1}{n} + \cdots + a_{m-1} \frac{1}{n^{m-1}} + a_m \frac{1}{n^m}\right)}{b_0 + b_1 \frac{1}{n} + \cdots + b_{k-1} \frac{1}{n^{k-1}} + b_k \frac{1}{n^k}}.
\]

Consider three cases:

1. If \(m < k\), then \(n^{m-k} \to 0\) as \(n \to \infty\), so the limit is \(0\).

2. If \(m = k\), then the highest degree terms dominate. As \(n \to \infty\), the terms involving \(\frac{1}{n}\) tend to \(0\), so the limit is \(\frac{a_0}{b_0}\).

3. If \(m > k\), then \(n^{m-k} \to \infty\) as \(n \to \infty\), so the limit is \(\infty\).

Therefore,
\[
\lim_{n \to \infty} \frac{a_0 n^m + a_1 n^{m-1} + \cdots + a_{m-1} n + a_m}{b_0 n^k + b_1 n^{k-1} + \cdots + b_{k-1} n + b_k} = 
\begin{cases} 
\frac{a_0}{b_0}, & \text{if } m = k, \\
0, & \text{if } m < k, \\
\infty, & \text{if } m > k.
\end{cases}
\]
\end{proof}
%
\begin{exercise}
Prove that $\lim_{n \to \infty} \frac{1}{n^k} = 0$ for any constant $k > 0$.
\end{exercise}
\begin{proof}
To show $\lim_{n \to \infty} \frac{1}{n^k} = 0$, we need to prove that for any $\varepsilon > 0$, there exists a positive integer $N$ such that for all $n > N$, $\left|\frac{1}{n^k} - 0\right| < \varepsilon$.

Let $\varepsilon > 0$ be given. We need to find $N$ such that
\[
\left|\frac{1}{n^k} - 0\right| < \varepsilon \iff \frac{1}{n^k} < \varepsilon \iff n^k > \frac{1}{\varepsilon}.
\]
Taking the $k$-th root, we get
\[
n > \left(\frac{1}{\varepsilon}\right)^{\frac{1}{k}}.
\]
Choose $N = \max\left\{1, \left(\frac{1}{\varepsilon}\right)^{\frac{1}{k}}  \right\}$. Then for all $n > N$, it follows that
\[
n^k > \left(\frac{1}{\varepsilon}\right) \implies \frac{1}{n^k} < \varepsilon,
\]
which completes the proof that $\lim_{n \to \infty} \frac{1}{n^k} = 0$.
\end{proof}
\begin{remark}
The choice of $N = \max\left\{1,  \left(\frac{1}{\varepsilon}\right)^{\frac{1}{k}}  \right\}$ ensures that $N$ is a positive integer.
\end{remark}
%
\begin{exercise}
Prove that $\lim_{n \to \infty} \sqrt[n]{a} = 1$ for any constant $a > 1$.
\end{exercise}
\begin{proof}
To show $\lim_{n \to \infty} \sqrt[n]{a} = 1$, we need to prove that for any $\varepsilon > 0$, there exists a positive integer $N$ such that for all $n > N$, $\left|\sqrt[n]{a} - 1\right| < \varepsilon$.

Let $\varepsilon > 0$ be given. We need to find $N$ such that
\[
\left|\sqrt[n]{a} - 1\right| < \varepsilon \iff \sqrt[n]{a} - 1 < \varepsilon \iff a^{\frac{1}{n}} < 1 + \varepsilon \iff \log_{a} a^{\frac{1}{n}} < \log_{a}(1 + \varepsilon) \iff \frac{1}{n} < \log_{a}(1 + \varepsilon).
\]
Therefore, we need
\[
n > \frac{1}{\log_{a}(1 + \varepsilon)}.
\]
Choose \(N \geq \frac{1}{\log_{a}(1 + \varepsilon)}\) and \(N \in \mathbb{Z}^+\). Then for all \(n > N\), it follows that
\[
n > \frac{1}{\log_{a}(1 + \varepsilon)} \implies \frac{1}{n} < \log_{a}(1 + \varepsilon) \implies a^{\frac{1}{n}} < 1 + \varepsilon \implies \sqrt[n]{a} - 1 < \varepsilon,
\]
which completes the proof that $\lim_{n \to \infty} \sqrt[n]{a} = 1$.
\end{proof}
%

\begin{exercise}
    Suppose \(\lim_{n \to \infty} a_n = a\), \(\lim_{n \to \infty} b_n = b\), and \(a < b\). Prove that there exists a positive integer \(N\) such that for all \(n > N\), \(a_n < b_n\).
\end{exercise}
\begin{proof}
    Since \(\lim_{n \to \infty} a_n = a\) and \(\lim_{n \to \infty} b_n = b\), and \(a < b\), take \(\epsilon = \frac{b - a}{2} > 0\). There exist positive integers \(N_1\) and \(N_2\) such that:
    
    For \(n > N_1\),
    \[
    |a_n - a| < \frac{b - a}{2},
    \]
    which implies
    \[
    \frac{3a - b}{2} < a_n < \frac{a + b}{2}.
    \]
    
    For \(n > N_2\),
    \[
    |b_n - b| < \frac{b - a}{2},
    \]
    which implies
    \[
    \frac{a + b}{2} < b_n < \frac{3b - a}{2}.
    \]
    
    Let \(N = \max\{N_1, N_2\}\). Then for \(n > N\), we have:
    \[
    a_n < \frac{a + b}{2} \quad \text{and} \quad \frac{a + b}{2} < b_n,
    \]
    which implies \(a_n < b_n\).
\end{proof}

%
\begin{exercise}
Prove the following sequence limits using the \(\varepsilon\)-\(N\) definition:

1. $\lim_{n \rightarrow \infty}(-1)^{n} \frac{1}{n^{3}}=0$;

2. $\lim_{n \rightarrow \infty} \frac{\sqrt[3]{n^{2}} \sin n!}{(n+1)^{2}}=0$;

3. $\lim_{n \rightarrow \infty}(\sqrt{n+1}-\sqrt{n})=0$;

4. $\lim_{n \rightarrow \infty} \frac{n}{100+n}=1$;

5. $\lim_{n \rightarrow \infty} \frac{n}{2n+1}=\frac{1}{2}$.
\end{exercise}
\begin{solution}

1. We need to show that for any \(\varepsilon > 0\), there exists \(N\) such that for all \(n > N\),
\[
\left| (-1)^{n} \frac{1}{n^{3}} - 0 \right| < \varepsilon.
\]
Since \(\left| (-1)^{n} \frac{1}{n^{3}} \right| = \frac{1}{n^{3}}\), we require
\[
\frac{1}{n^{3}} < \varepsilon \quad \Leftrightarrow \quad n^{3} > \frac{1}{\varepsilon}.
\]
Let \(N = \left( \frac{1}{\varepsilon} \right)^{1/3}\). Then for all \(n > N\), we have
\[
\frac{1}{n^{3}} < \varepsilon.
\]
Thus, \(\lim_{n \to \infty} (-1)^{n} \frac{1}{n^{3}} = 0\).

2. We need to show that for any \(\varepsilon > 0\), there exists \(N\) such that for all \(n > N\),
\[
\left| \frac{\sqrt[3]{n^{2}} \sin n!}{(n+1)^{2}} - 0 \right| < \varepsilon.
\]
Since \(\left| \frac{\sqrt[3]{n^{2}} \sin n!}{(n+1)^{2}} \right| \leq \frac{\sqrt[3]{n^{2}}}{(n+1)^{2}} \leq \frac{n^{2/3}}{n^2}\), we have
\[
\frac{n^{2/3}}{n^2} = \frac{1}{n^{4/3}} \leq \frac{1}{n}.
\]
We require
\[
\frac{1}{n} < \varepsilon \quad \Leftrightarrow \quad n > \frac{1}{\varepsilon}.
\]
Let \(N = \left( \frac{1}{\varepsilon} \right)\). Then for all \(n > N\), we have
\[
\frac{1}{n^{4/3}} < \frac{1}{n} < \varepsilon.
\]
Thus, \(\lim_{n \to \infty} \frac{\sqrt[3]{n^{2}} \sin n!}{(n+1)^{2}} = 0\).

3. We need to show that for any \(\varepsilon > 0\), there exists \(N\) such that for all \(n > N\),
\[
\left| \sqrt{n+1} - \sqrt{n} - 0 \right| < \varepsilon.
\]
Using the rationalization technique,
\[
\sqrt{n+1} - \sqrt{n} = \frac{(\sqrt{n+1} - \sqrt{n})(\sqrt{n+1} + \sqrt{n})}{\sqrt{n+1} + \sqrt{n}} = \frac{1}{\sqrt{n+1} + \sqrt{n}}.
\]
Since \(\sqrt{n+1} + \sqrt{n} > \sqrt{n}\), we have
\[
\frac{1}{\sqrt{n+1} + \sqrt{n}} < \frac{1}{\sqrt{n}}.
\]
We require
\[
\frac{1}{\sqrt{n}} < \varepsilon \quad \Leftrightarrow \quad n > \frac{1}{\varepsilon^2}.
\]
Let \(N = \frac{1}{\varepsilon^2}\). Then for all \(n > N\), we have
\[
\frac{1}{\sqrt{n}} < \varepsilon.
\]
Thus, \(\lim_{n \to \infty} (\sqrt{n+1} - \sqrt{n}) = 0\).

4. We need to show that for any \(\varepsilon > 0\), there exists \(N\) such that for all \(n > N\),
\[
\left| \frac{n}{100+n} - 1 \right| < \varepsilon.
\]
Since
\[
\left| \frac{n}{100+n} - 1 \right| = \left| \frac{n - (100+n)}{100+n} \right| = \left| \frac{-100}{100+n} \right| = \frac{100}{100+n} < \frac{100}{n},
\]
we require
\[
\frac{100}{n} < \varepsilon \quad \Leftrightarrow \quad 100 < \varepsilon n.
\]
Thus,
\[
\frac{100}{\varepsilon} < n \quad \Leftrightarrow \quad n > \frac{100}{\varepsilon}.
\]
Let \(N = \frac{100}{\varepsilon}\). Then for all \(n > N\), we have
\[
\frac{100}{100+n}< \frac{100}{n} < \varepsilon.
\]
Thus, \(\lim_{n \to \infty} \frac{n}{100+n} = 1\).

5. We need to show that for any \(\varepsilon > 0\), there exists \(N\) such that for all \(n > N\),
\[
\left| \frac{n}{2n+1} - \frac{1}{2} \right| < \varepsilon.
\]
Since
\[
\left| \frac{n}{2n+1} - \frac{1}{2} \right| = \left| \frac{2n - (2n+1)}{2(2n+1)} \right| = \left| \frac{-1}{2(2n+1)} \right| = \frac{1}{2(2n+1)} < \frac{1}{4n} < \frac{1}{n},
\]
we require
\[
\frac{1}{n} < \varepsilon \quad \Leftrightarrow \quad 1 < \varepsilon n.
\]
Thus,
\[
\frac{1}{\varepsilon} < n \quad \Leftrightarrow \quad n > \frac{1}{\varepsilon}.
\]
Let \(N = \frac{1}{\varepsilon} \). Then for all \(n > N\), we have
\[
\frac{1}{2(2n+1)}< \frac{1}{n}< \varepsilon.
\]
Thus, \(\lim_{n \to \infty} \frac{n}{2n+1} = \frac{1}{2}\).

\end{solution}

\begin{exercise}
Compute the following limits:
\begin{enumerate}
    \item $\lim_{n \to \infty} \frac{(-2)^n + 3^n}{(-2)^{n+1} + 3^{n+1}}$
    \item $\lim_{n \to \infty} \frac{n^{5/2} - n + 6}{2n^{5/2} + 2n^2 - 7}$
    \item $\lim_{n \to \infty} \frac{1 + a + a^2 + \cdots + a^n}{1 + b + b^2 + \cdots + b^n} \quad (|a|<1, |b|<1)$
    \item $\lim_{n \to \infty} \left( \frac{1}{n^2} + \frac{2}{n^2} + \cdots + \frac{n-1}{n^2} \right)$
    \item $\lim_{n \to \infty} \left[ \frac{1^2}{n^3} + \frac{2^2}{n^3} + \cdots + \frac{(n-1)^2}{n^3} \right]$
    \item $\lim_{n \to \infty} \left[ \frac{1}{1 \times 2} + \frac{1}{2 \times 3} + \frac{1}{3 \times 4} + \cdots + \frac{1}{n(n+1)} \right]$
\end{enumerate}
\end{exercise}
\begin{solution}
\begin{enumerate}
    \item $\lim_{n \to \infty} \frac{(-2)^n + 3^n}{(-2)^{n+1} + 3^{n+1}}$
    \begin{proof}
    The dominant term in the numerator and denominator is $3^n$ and $3^{n+1}$, respectively. Thus,
    \[
    \lim_{n \to \infty} \frac{(-2)^n + 3^n}{(-2)^{n+1} + 3^{n+1}} = \lim_{n \to \infty} \frac{3^n(1 + (-2/3)^n)}{3^{n+1}(1 + (-2/3)^{n+1})} = \lim_{n \to \infty} \frac{1 + 0}{3} = \frac{1}{3}.
    \]
    \end{proof}
    
    \item $\lim_{n \to \infty} \frac{n^{5/2} - n + 6}{2n^{5/2} + 2n^2 - 7}$
    \begin{proof}
    The dominant term in both the numerator and the denominator is $n^{5/2}$. Thus,
    \[
    \lim_{n \to \infty} \frac{n^{5/2} - n + 6}{2n^{5/2} + 2n^2 - 7} = \lim_{n \to \infty} \frac{1 - n^{-3/2} + 6n^{-5/2}}{2 + 2n^{-1/2} - 7n^{-5/2}} = \frac{1}{2}.
    \]
    \end{proof}

    \item $\lim_{n \to \infty} \frac{1 + a + a^2 + \cdots + a^n}{1 + b + b^2 + \cdots + b^n} \quad (|a|<1, |b|<1)$
    \begin{proof}
    The sums are geometric series. The sums can be written as:
    \[
    \frac{1 + a + a^2 + \cdots + a^n}{1 + b + b^2 + \cdots + b^n} = \frac{\frac{1-a^{n+1}}{1-a}}{\frac{1-b^{n+1}}{1-b}} = \frac{1-a^{n+1}}{1-b^{n+1}} \cdot \frac{1-b}{1-a}.
    \]
    As $n \to \infty$, $a^{n+1} \to 0$ and $b^{n+1} \to 0$. Therefore,
    \[
    \lim_{n \to \infty} \frac{1-a^{n+1}}{1-b^{n+1}} \cdot \frac{1-b}{1-a} = \frac{1 \cdot (1-b)}{1 \cdot (1-a)} = \frac{1-b}{1-a}.
    \]
    \end{proof}

    \item $\lim_{n \to \infty} \left( \frac{1}{n^2} + \frac{2}{n^2} + \cdots + \frac{n-1}{n^2} \right)$
    \begin{proof}
    This is an arithmetic series where each term can be written as $\frac{k}{n^2}$ for $k = 1, 2, \ldots, n-1$. Thus,
    \[
    \lim_{n \to \infty} \left( \frac{1}{n^2} + \frac{2}{n^2} + \cdots + \frac{n-1}{n^2} \right) = \lim_{n \to \infty} \frac{1 + 2 + \cdots + (n-1)}{n^2} = \lim_{n \to \infty} \frac{\frac{(n-1)n}{2}}{n^2} = \lim_{n \to \infty} \frac{n-1}{2n} = \frac{1}{2}.
    \]
    \end{proof}

    \item $\lim_{n \to \infty} \left[ \frac{1^2}{n^3} + \frac{2^2}{n^3} + \cdots + \frac{(n-1)^2}{n^3} \right]$
    \begin{proof}
    This is a sum of squares. We can factor out $\frac{1}{n^3}$:
    \[
    \lim_{n \to \infty} \left[ \frac{1^2}{n^3} + \frac{2^2}{n^3} + \cdots + \frac{(n-1)^2}{n^3} \right] = \lim_{n \to \infty} \frac{1^2 + 2^2 + \cdots + (n-1)^2}{n^3}.
    \]
    Using the sum of squares formula $\sum_{k=1}^{n-1} k^2 = \frac{(n-1)n(2n-1)}{6}$,
    \[
    \lim_{n \to \infty} \frac{\frac{(n-1)n(2n-1)}{6}}{n^3} = \lim_{n \to \infty} \frac{(n-1)n(2n-1)}{6n^3} = \lim_{n \to \infty} \frac{(n-1)(2n-1)}{6n^2} = \frac{1}{3}.
    \]
    \end{proof}

    \item $\lim_{n \to \infty} \left[ \frac{1}{1 \times 2} + \frac{1}{2 \times 3} + \frac{1}{3 \times 4} + \cdots + \frac{1}{n(n+1)} \right]$
    \begin{proof}
    Each term in the series can be simplified as:
    \[
    \frac{1}{k(k+1)} = \frac{1}{k} - \frac{1}{k+1}.
    \]
    Thus, the series becomes a telescoping series:
    \[
    \sum_{k=1}^{n} \left( \frac{1}{k} - \frac{1}{k+1} \right) = \left( 1 - \frac{1}{2} \right) + \left( \frac{1}{2} - \frac{1}{3} \right) + \cdots + \left( \frac{1}{n} - \frac{1}{n+1} \right).
    \]
    All intermediate terms cancel out, leaving:
    \[
    \lim_{n \to \infty} \left( 1 - \frac{1}{n+1} \right) = 1.
    \]
    \end{proof}
\end{enumerate}
\end{solution}


\begin{exercise}
Compute the following limits using the Squeeze Theorem:
\begin{enumerate}
    \item $\lim_{n \to \infty}\left(\frac{\sqrt{1 \times 2}}{n^{2}+1}+\frac{\sqrt{2 \times 3}}{n^{2}+2}+\cdots+\frac{\sqrt{n(n+1)}}{n^{2}+n}\right)$
    \item $\lim_{n \to \infty} \sqrt[n]{a^{n}+b^{n}} \quad(a>0, b>0)$
    \item $\lim_{n \to \infty} \sqrt[n]{n^{p}+n^{q}} \quad(p, q \text{ are positive integers})$
    \item $\lim_{n \to \infty}\left[(n+1)^{\alpha}-n^{\alpha}\right] \quad(0<\alpha<1 \text{ is a constant})$
    \item $\lim_{n \to \infty} \frac{1}{\sqrt{n!}}$
    \item $\lim_{n \to \infty} \sin(\pi \sqrt{n^2 + n} + 1)$
\end{enumerate}
\end{exercise}
\begin{solution}
\begin{enumerate}
    \item $\lim_{n \to \infty}\left(\frac{\sqrt{1 \times 2}}{n^{2}+1}+\frac{\sqrt{2 \times 3}}{n^{2}+2}+\cdots+\frac{\sqrt{n(n+1)}}{n^{2}+n}\right)$
    \begin{proof}
    Each term $\frac{\sqrt{k(k+1)}}{n^2 + k}$ is bounded above by $\frac{k+1}{n^2}$ and below by $\frac{k}{n^2}$. Thus,
    \[
    \frac{k}{n^2} \leq \frac{\sqrt{k(k+1)}}{n^2 + k} \leq \frac{k+1}{n^2}.
    \]
    Summing these inequalities from $k=1$ to $k=n$ gives:
    \[
    \sum_{k=1}^{n} \frac{k}{n^2} \leq \sum_{k=1}^{n} \frac{\sqrt{k(k+1)}}{n^2 + k} \leq \sum_{k=1}^{n} \frac{k+1}{n^2}.
    \]
    The sums of the bounds are:
    \[
    \sum_{k=1}^{n} \frac{k}{n^2} = \frac{1}{n^2} \sum_{k=1}^{n} k = \frac{1}{n^2} \cdot \frac{n(n+1)}{2} = \frac{n+1}{2n} \to \frac{1}{2} \text{ as } n \to \infty,
    \]
    and
    \[
    \sum_{k=1}^{n} \frac{k+1}{n^2} = \frac{1}{n^2} \sum_{k=1}^{n} (k+1) = \frac{1}{n^2} \left(\frac{n(n+1)}{2} + n\right) = \frac{n+1}{2n} + \frac{1}{n} \to \frac{1}{2} \text{ as } n \to \infty.
    \]
    Therefore, by the Squeeze Theorem,
    \[
    \lim_{n \to \infty} \left(\frac{\sqrt{1 \times 2}}{n^2 + 1} + \frac{\sqrt{2 \times 3}}{n^2 + 2} + \cdots + \frac{\sqrt{n(n+1)}}{n^2 + n}\right) = \frac{1}{2}.
    \]
    \end{proof}

    \item $\lim_{n \to \infty} \sqrt[n]{a^{n}+b^{n}} \quad(a>0, b>0)$
    \begin{proof}
    Assume without loss of generality that $a \geq b$. Then,
    \[
    a \leq \sqrt[n]{a^n + b^n} \leq \sqrt[n]{2a^n} = a \sqrt[n]{2}.
    \]
    As $n \to \infty$, $\sqrt[n]{2} \to 1$. Therefore, by the Squeeze Theorem,
    \[
    \lim_{n \to \infty} \sqrt[n]{a^n + b^n} = a.
    \]
    \end{proof}

    \item $\lim_{n \to \infty} \sqrt[n]{n^{p}+n^{q}} \quad(p, q \text{ are positive integers})$
    \begin{proof}
    Assume without loss of generality that $p \geq q$. Then,
    \[
    n^{p/n} \leq \sqrt[n]{n^p + n^q} \leq \sqrt[n]{2n^p} = n^{p/n} \sqrt[n]{2}.
    \]
    As $n \to \infty$, $n^{p/n} \to 1$ and $\sqrt[n]{2} \to 1$. Therefore, by the Squeeze Theorem,
    \[
    \lim_{n \to \infty} \sqrt[n]{n^p + n^q} = 1.
    \]
    \end{proof}

    \item $\lim_{n \to \infty}\left[(n+1)^{\alpha}-n^{\alpha}\right] \quad(0<\alpha<1 \text{ is a constant})$
    \begin{proof}
    Consider the function $f(x) = x^{\alpha}$. Using the Mean Value Theorem, there exists some $\xi \in (n, n+1)$ such that
    \[
    (n+1)^{\alpha} - n^{\alpha} = f'( \xi ) = \alpha \xi^{\alpha-1}.
    \]
    Since $n < \xi < n+1$, we have
    \[
    n^{\alpha-1} < \xi^{\alpha-1} < (n+1)^{\alpha-1}.
    \]
    Thus,
    \[
    \alpha n^{\alpha-1} < \alpha \xi^{\alpha-1} < \alpha (n+1)^{\alpha-1}.
    \]
    As $n \to \infty$, $\alpha n^{\alpha-1} \to 0$. Therefore, by the Squeeze Theorem,
    \[
    \lim_{n \to \infty} \alpha \xi^{\alpha-1} = 0.
    \]
    Hence,
    \[
    \lim_{n \to \infty} \left[(n+1)^{\alpha} - n^{\alpha}\right] = 0.
    \]
    \end{proof}

    \item $\lim_{n \to \infty} \frac{1}{\sqrt{n!}}$
    \begin{proof}
    Using Stirling's approximation $n! \approx \sqrt{2\pi n} \left(\frac{n}{e}\right)^n$, we get
    \[
    \frac{1}{\sqrt{n!}} \approx \frac{1}{\sqrt{\sqrt{2\pi n} \left(\frac{n}{e}\right)^n}} = \frac{1}{\sqrt{(2\pi n)^{1/2} \cdot \left(\frac{n}{e}\right)^n}}.
    \]
    Simplifying, we get
    \[
    \frac{1}{\sqrt{(2\pi n)^{1/2}} \cdot n^{n/2} \cdot e^{-n/2}} = \frac{e^{n/2}}{(2\pi n)^{1/4} \cdot n^{n/2}}.
    \]
    As $n \to \infty$, the denominator grows much faster than the numerator, so the limit is
    \[
    \lim_{n \to \infty} \frac{1}{\sqrt{n!}} = 0.
    \]
    \end{proof}

    \item $\lim_{n \to \infty} \sin(\pi \sqrt{n^2 + n} + 1)$
    \begin{proof}
    Consider the expression inside the sine function:
    \[
    \pi \sqrt{n^2 + n} + 1 = \pi n \sqrt{1 + \frac{1}{n}} + 1.
    \]
    As $n \to \infty$, $\sqrt{1 + \frac{1}{n}} \to 1$. Therefore,
    \[
    \pi n \sqrt{1 + \frac{1}{n}} + 1 \approx \pi n + \frac{\pi}{2} + 1.
    \]
    Hence,
    \[
    \sin(\pi \sqrt{n^2 + n} + 1) \approx \sin\left(\pi n + \frac{\pi}{2} + 1\right).
    \]
    Since $\sin(\pi n + \theta) = \sin(\theta)$ for any integer $n$,
    \[
    \sin\left(\pi n + \frac{\pi}{2} + 1\right) = \sin\left(\frac{\pi}{2} + 1\right) \to 0 \text{ as } n \to \infty.
    \]
    \end{proof}
\end{enumerate}
\end{solution}

\begin{exercise}
Let $0 \leq a < b$, $x_1 = a$, $y_1 = b$, and
\[
x_{n+1} = \frac{x_n + y_n}{2}, \quad y_{n+1} = \sqrt{x_n y_n}.
\]
Prove that the sequences $\{x_n\}$ and $\{y_n\}$ are convergent and find their limits.
\end{exercise}
\begin{solution}
To prove the convergence of the sequences $\{x_n\}$ and $\{y_n\}$, we use the Arithmetic-Geometric Mean (AGM) inequality.

\begin{proof}
Given the initial conditions, $x_1 = a > 0$ and $y_1 = b > 0$. We need to prove the following properties of the sequences:
\[
x_{n+1} = \frac{x_n + y_n}{2} \quad \text{and} \quad y_{n+1} = \sqrt{x_n y_n}.
\]

First, we apply the AGM inequality:
\[
\sqrt{x_n y_n} \leq \frac{x_n + y_n}{2}.
\]

Therefore, we have:
\[
y_{n+1} = \sqrt{x_n y_n} \leq x_{n+1} = \frac{x_n + y_n}{2}.
\]

Next, we show that $\{x_n\}$ is decreasing:
\[
x_{n+1} = \frac{x_n + y_n}{2} \leq x_n,
\]
since $y_n \leq x_n$. Subtracting $x_n$ from both sides, we get:
\[
x_{n+1} - x_n = \frac{y_n - x_n}{2} \leq 0.
\]
Thus, $\{x_n\}$ is a decreasing sequence. Since $x_{n+1} \geq y_{n+1}$ and $y_{n+1} \geq 0$, $\{x_n\}$ is bounded below by $y_{n+1}$, which is non-negative.

Next, we show that $\{y_n\}$ is increasing:
\[
y_{n+1} = \sqrt{x_n y_n} \geq y_n,
\]
since $x_n \geq y_n$. Subtracting $y_n$ from both sides, we get:
\[
y_{n+1} - y_n = \sqrt{x_n y_n} - y_n \geq 0.
\]
Thus, $\{y_n\}$ is an increasing sequence. Since $y_{n+1} \leq x_{n+1}$ and $x_{n+1}$ is bounded above by $x_1$, $\{y_n\}$ is bounded above.

By the Monotone Convergence Theorem, since $\{x_n\}$ is decreasing and bounded below, it converges. Similarly, since $\{y_n\}$ is increasing and bounded above, it also converges.

Let $\lim_{n \to \infty} x_n = \alpha$ and $\lim_{n \to \infty} y_n = \beta$. From the recursion relations, we have:
\[
\alpha = \lim_{n \to \infty} x_{n+1} = \lim_{n \to \infty} \frac{x_n + y_n}{2} = \frac{\alpha + \beta}{2},
\]
and
\[
\beta = \lim_{n \to \infty} y_{n+1} = \lim_{n \to \infty} \sqrt{x_n y_n} = \sqrt{\alpha \beta}.
\]

Solving the first equation for $\alpha$, we get:
\[
2\alpha = \alpha + \beta \implies \alpha = \beta.
\]

Substituting $\alpha = \beta$ into the second equation, we get:
\[
\beta = \sqrt{\alpha \beta} = \sqrt{\alpha^2} = \alpha.
\]

Therefore, the limits of both sequences are equal, and we have:
\[
\alpha = \beta.
\]

Thus, the sequences $\{x_n\}$ and $\{y_n\}$ converge to the same limit, which is $\sqrt{ab}$.
\end{proof}
\end{solution}

\begin{exercise}
Use the definition of limit to prove that $\lim_{n \to \infty} \left[\left(1+a_{n}\right)^{-1}\right]=\frac{1}{2}$ if $\lim_{n \to \infty} a_{n}=1$.
\end{exercise}

\begin{solution}
We will use the definition of a limit to prove this statement.

Given: $\lim_{n \to \infty} a_{n}=1$

To prove: $\lim_{n \to \infty} \left[\left(1+a_{n}\right)^{-1}\right]=\frac{1}{2}$

By the definition of limit, we need to show that for any $\varepsilon > 0$, there exists an $N \in \mathbb{N}$ such that for all $n \geq N$,

\[
\left|\left(1+a_{n}\right)^{-1} - \frac{1}{2}\right| < \varepsilon
\]

Let $\varepsilon > 0$ be given.

Since $\lim_{n \to \infty} a_{n}=1$, for any $\delta > 0$, there exists an $M \in \mathbb{N}$ such that for all $n \geq M$,

\[
|a_{n} - 1| < \delta
\]

Let's choose $\delta = \min\left\{\frac{1}{2}, \frac{2\varepsilon}{1+2\varepsilon}\right\}$.

Now, for $n \geq M$:

\begin{align*}
\left|\left(1+a_{n}\right)^{-1} - \frac{1}{2}\right| &= \left|\frac{1}{1+a_{n}} - \frac{1}{2}\right| \\[6pt]
&= \left|\frac{2-(1+a_{n})}{2(1+a_{n})}\right| \\[6pt]
&= \left|\frac{1-a_{n}}{2(1+a_{n})}\right| \\[6pt]
&= \frac{|1-a_{n}|}{2|1+a_{n}|}
\end{align*}

We know that $|a_{n} - 1| < \delta \leq \frac{1}{2}$, so $\frac{1}{2} < a_{n} < \frac{3}{2}$.
Therefore, $\frac{3}{2} < 1+a_{n} < \frac{5}{2}$, and $\frac{2}{5} < \frac{1}{1+a_{n}} < \frac{2}{3}$.

Now,

\begin{align*}
\frac{|1-a_{n}|}{2|1+a_{n}|} &< \frac{\delta}{2 \cdot \frac{5}{2}} \\[6pt]
&= \frac{\delta}{5} \\[6pt]
&\leq \frac{1}{5} \cdot \frac{2\varepsilon}{1+2\varepsilon} \\[6pt]
&< \varepsilon
\end{align*}

Therefore, for any $\varepsilon > 0$, there exists an $N = M$ such that for all $n \geq N$,

\[
\left|\left(1+a_{n}\right)^{-1} - \frac{1}{2}\right| < \varepsilon
\]

This proves that $\lim_{n \to \infty} \left[\left(1+a_{n}\right)^{-1}\right]=\frac{1}{2}$.
\end{solution}



\section{Limit of Function}



\section{Countability of Sets}
\begin{definition}[Countability of a Set]
A set $S$ is said to be countable if:
\begin{itemize}
    \item It is finite, or
    \item It is infinite and there exists a bijective function $f: \mathbb{N} \rightarrow S$, where $\mathbb{N}$ is the set of natural numbers.
\end{itemize}
In other words, a countable set is one whose elements can be put in a one-to-one correspondence with the natural numbers or a subset of the natural numbers.
\end{definition}



\begin{exercise}
    Prove that every infinite subset of a countable set is countable. 
\end{exercise}
\begin{proof}
    Let \( S \) be a countable infinite set. This means there exists a bijection \( f_S: \mathbb{N} \to S \). We can denote the elements of \( S \) as a sequence:
    \[
    S = \{s_1, s_2, s_3, \ldots\}
    \]
    where \( s_n = f_S(n) \).

    Now, let \( T \subseteq S \) be an infinite subset of \( S \). We need to prove that \( T \) is countable. To do this, we will construct a bijection \( f_T: \mathbb{N} \to T \).

    Since \( T \) is an infinite subset of \( S \), the elements of \( T \) can also be listed in a sequence. We construct this sequence by selecting elements from \( S \) that are in \( T \).

    1. \textbf{Construction of the sequence \(\{t_n\}\)}:
        \begin{itemize}
            \item Let \( t_1 \) be the first element of \( S \) that is in \( T \):
              \[
              t_1 = s_{n_1} \quad \text{where} \quad n_1 = \min \{ n \in \mathbb{N} \mid s_n \in T \}
              \]

            \item Let \( t_2 \) be the second element of \( S \) that is in \( T \):
              \[
              t_2 = s_{n_2} \quad \text{where} \quad n_2 = \min \{ n \in \mathbb{N} \mid n > n_1 \text{ and } s_n \in T \}
              \]

            \item Generally, let \( t_k \) be the \( k \)-th element of \( S \) that is in \( T \):
              \[
              t_k = s_{n_k} \quad \text{where} \quad n_k = \min \{ n \in \mathbb{N} \mid n > n_{k-1} \text{ and } s_n \in T \}
              \]
        \end{itemize}

    2. \textbf{Define the bijection \( f_T: \mathbb{N} \to T \)}:
        \begin{itemize}
            \item Let \( f_T(n) = t_n \). By construction, \( f_T \) maps natural numbers \( n \) to distinct elements \( t_n \in T \).
        \end{itemize}

    Since \( T \) is an infinite subset of \( S \), and each \( t_k \) is an element of \( T \), the function \( f_T \) is a bijection from \(\mathbb{N}\) to \( T \).

    Therefore, \( T \) is countable.

    In conclusion, any infinite subset of a countable set is also countable.
\end{proof}

\begin{exercise}
    Does the existence of a countable infinite subset of a set imply that the set itself is countable? If so, prove it, else give a counter example.
\end{exercise}
\begin{solution}
    The answer is No.

    Consider the set of all real numbers \(\mathbb{R}\). We know that \(\mathbb{R}\) is uncountable.

    However, \(\mathbb{R}\) contains the set of all rational numbers \(\mathbb{Q}\) as a subset. The set of rational numbers \(\mathbb{Q}\) is a countable infinite subset of \(\mathbb{R}\).
    
    Despite the fact that \(\mathbb{R}\) contains the countable infinite subset \(\mathbb{Q}\), \(\mathbb{R}\) itself is uncountable. This demonstrates that the existence of a countable infinite subset of a set does not imply that the set itself is countable.
\end{solution}


\begin{exercise}
    Prove that every set that contains an uncountable subset is uncountable. Thus, conclude that $\mathbb{R}$ is uncountable if $[0,1]$ is uncountable.
\end{exercise}
\begin{proof}
    Suppose $A$ is a set that contains an uncountable subset $B$. We will prove that $A$ is uncountable by contradiction.

    Assume, for the sake of contradiction, that $A$ is countable. This means $A$ is either finite or countably infinite.

    \begin{itemize}
        \item If $A$ is finite, then $B$ must also be finite since $B \subseteq A$. This contradicts the assumption that $B$ is uncountable.
        \item If $A$ is countably infinite, then there exists a bijection $f: \mathbb{N} \to A$. Since $B \subseteq A$, we can map elements of $\mathbb{N}$ to elements of $B$ using $f$. This implies that $B$ is countable because a subset of a countably infinite set is at most countably infinite. This contradicts the assumption that $B$ is uncountable.
    \end{itemize}

    Therefore, our assumption that $A$ is countable must be false. Hence, $A$ is uncountable.

    To conclude, since $[0,1] \subseteq \mathbb{R}$ and $[0,1]$ is uncountable, it follows that $\mathbb{R}$ is uncountable.
\end{proof}

\begin{exercise}
Prove that $\mathbb{R} \times \mathbb{R}$ has the same cardinality as $\mathbb{R}$ using decimal expansions.
\end{exercise}

\begin{solution}
We will prove that $\mathbb{R} \times \mathbb{R}$ has the same cardinality as $\mathbb{R}$ by constructing a bijective function $f: \mathbb{R} \times \mathbb{R} \to \mathbb{R}$ using decimal expansions.

Let $(x, y) \in \mathbb{R} \times \mathbb{R}$. We can represent $x$ and $y$ using their decimal expansions:

\[
x = \pm a_0.a_1a_2a_3\ldots
\]
\[
y = \pm b_0.b_1b_2b_3\ldots
\]

where $a_0, b_0 \in \mathbb{N}$ and $a_i, b_i \in \{0, 1, 2, \ldots, 9\}$ for $i \geq 1$.

Now, we define $f(x, y)$ as follows:

\[
f(x, y) = \pm c_0.c_1c_2c_3c_4c_5c_6\ldots
\]

where:
\begin{itemize}
    \item The sign is positive if both $x$ and $y$ are non-negative, negative otherwise.
    \item $c_0 = a_0$
    \item $c_{2i-1} = a_i$ for $i \geq 1$
    \item $c_{2i} = b_i$ for $i \geq 0$
\end{itemize}

In other words, we interleave the digits of the decimal expansions of $x$ and $y$, starting with the fractional part of $y$.

To prove that $f$ is bijective:

1) Injective: For any two distinct pairs $(x_1, y_1)$ and $(x_2, y_2)$, $f(x_1, y_1) \neq f(x_2, y_2)$ because their decimal expansions will differ in at least one digit.

2) Surjective: For any $r \in \mathbb{R}$, we can construct a pair $(x, y)$ such that $f(x, y) = r$ by separating the even and odd-indexed digits in the decimal expansion of $r$.

Therefore, $f$ is a bijection between $\mathbb{R} \times \mathbb{R}$ and $\mathbb{R}$, proving that these sets have the same cardinality.

Note: This proof assumes the decimal expansions are unique. To make it fully rigorous, we need to address the issue of numbers with two decimal representations (e.g., $0.999\ldots = 1.000\ldots$). This can be done by choosing a canonical representation for such numbers.
\end{solution}

\section{Differentiation}

\subsection{Differentiation Rules}
\subsubsection{Basic Functions}

\begin{table}[H]
\centering
\begin{tabular}{@{}lll@{}}
\toprule
Function & Derivative & Equivalent Forms \\
\midrule
$c$ (constant) & $(c)' = 0$ & \\
$x^n$ & $(x^n)' = nx^{n-1}$ & \\
$e^x$ & $(e^x)' = e^x$ & \\
$a^x$ & $(a^x)' = a^x \ln(a)$ & \\
$\ln(x)$ & $(\ln(x))' = \frac{1}{x}$ & \\
$\log_a(x)$ & $(\log_a(x))' = \frac{1}{x \ln(a)}$ & $\frac{\log_e(x)}{\log_e(a)}$ \\
$\sqrt{x}$ & $(\sqrt{x})' = \frac{1}{2\sqrt{x}}$ & $\frac{1}{2}x^{-1/2}$ \\
$x^{1/n}$ & $(x^{1/n})' = \frac{1}{n}x^{(1-n)/n}$ & \\
$|x|$ & $(|x|)' = \frac{x}{|x|}$ & $\text{sgn}(x)$ \\
$f^{-1}(x)$ & $(f^{-1}(x))' = \frac{1}{f'(f^{-1}(x))}$ & \\
\bottomrule
\end{tabular}
\caption{Differentiation Rules for Basic Functions}
\label{tab:basic_derivatives}
\end{table}

\subsubsection{Composite Functions}

\begin{table}[H]
\centering
\begin{tabular}{@{}ll@{}}
\toprule
Function & Derivative \\
\midrule
$u(x)v(x)$ & $(uv)' = u'v + uv'$ \\
$\frac{u(x)}{v(x)}$ & $(\frac{u}{v})' = \frac{u'v - uv'}{v^2}$ \\
$f(g(x))$ & $(f(g(x)))' = f'(g(x)) \cdot g'(x)$ \\
\bottomrule
\end{tabular}
\caption{Differentiation Rules for Composite Functions}
\label{tab:composite_derivatives}
\end{table}

\subsubsection{Trigonometric Functions}

\begin{table}[H]
\centering
\begin{tabular}{@{}lll@{}}
\toprule
Function & Derivative & Equivalent Forms \\
\midrule
$\sin x$ & $(\sin x)' = \cos x$ & \\
$\cos x$ & $(\cos x)' = -\sin x$ & \\
$\tan x$ & $(\tan x)' = \sec^2 x$ & $1 + \tan^2 x$ \\
$\cot x$ & $(\cot x)' = -\csc^2 x$ & $-(1 + \cot^2 x)$ \\
$\sec x$ & $(\sec x)' = \sec x \tan x$ & $\frac{\sin x}{\cos^2 x}$ \\
$\csc x$ & $(\csc x)' = -\csc x \cot x$ & $-\frac{\cos x}{\sin^2 x}$ \\
\bottomrule
\end{tabular}
\caption{Derivatives of Trigonometric Functions}
\label{tab:trig_derivatives}
\end{table}

\subsubsection{Inverse Trigonometric Functions}

\begin{table}[H]
\centering
\begin{tabular}{@{}lll@{}}
\toprule
Function & Derivative & Equivalent Forms \\
\midrule
$\arcsin x$ & $(\arcsin x)' = \frac{1}{\sqrt{1-x^2}}$ & $\frac{1}{\cos(\arcsin x)}$ \\
$\arccos x$ & $(\arccos x)' = -\frac{1}{\sqrt{1-x^2}}$ & $-\frac{1}{\sin(\arccos x)}$ \\
$\arctan x$ & $(\arctan x)' = \frac{1}{1+x^2}$ & $\cos^2(\arctan x)$ \\
$\text{arccot } x$ & $(\text{arccot } x)' = -\frac{1}{1+x^2}$ & $-\sin^2(\text{arccot } x)$ \\
$\text{arcsec } x$ & $(\text{arcsec } x)' = \frac{1}{|x|\sqrt{x^2-1}}$ & $\frac{\cos(\text{arcsec } x)}{x\sin(\text{arcsec } x)}$ \\
$\text{arccsc } x$ & $(\text{arccsc } x)' = -\frac{1}{|x|\sqrt{x^2-1}}$ & $-\frac{\sin(\text{arccsc } x)}{x\cos(\text{arccsc } x)}$ \\
\bottomrule
\end{tabular}
\caption{Derivatives of Inverse Trigonometric Functions}
\label{tab:inverse_trig_derivatives}
\end{table}

\subsubsection{Important Trigonometric Identities}

Here are some important properties of trigonometric functions that may help when integrate trigonometric functions, or doing trigonometric substitutions.
\begin{table}[H]
\centering
\begin{tabular}{@{}ll@{}}
\toprule
Identity & Formula \\
\midrule
Pythagorean & $\sin^2 x + \cos^2 x = 1$ \\
 & $\tan^2 x + 1 = \sec^2 x$ \\
 & $\cot^2 x + 1 = \csc^2 x$ \\
\midrule
Reciprocal & $\sec x = \frac{1}{\cos x}$, $\csc x = \frac{1}{\sin x}$ \\
\midrule
Quotient & $\tan x = \frac{\sin x}{\cos x}$, $\cot x = \frac{\cos x}{\sin x}$ \\
\midrule
Even-Odd & $\sin(-x) = -\sin x$, $\cos(-x) = \cos x$ \\
 & $\tan(-x) = -\tan x$, $\cot(-x) = -\cot x$ \\
\midrule
Periodicity & $\sin(x + 2\pi) = \sin x$, $\cos(x + 2\pi) = \cos x$ \\
 & $\tan(x + \pi) = \tan x$ \\
\bottomrule
\end{tabular}
\caption{Basic Trigonometric Identities}
\label{tab:basic_trig_identities}
\end{table}

\begin{table}[h]
\centering
\begin{tabular}{@{}ll@{}}
\toprule
Identity & Formula \\
\midrule
Sum & $\sin(A + B) = \sin A \cos B + \cos A \sin B$ \\
 & $\cos(A + B) = \cos A \cos B - \sin A \sin B$ \\
 & $\tan(A + B) = \frac{\tan A + \tan B}{1 - \tan A \tan B}$ \\
\midrule
Difference & $\sin(A - B) = \sin A \cos B - \cos A \sin B$ \\
 & $\cos(A - B) = \cos A \cos B + \sin A \sin B$ \\
 & $\tan(A - B) = \frac{\tan A - \tan B}{1 + \tan A \tan B}$ \\
\midrule
Double Angle & $\sin 2x = 2\sin x \cos x$ \\
 & $\cos 2x = \cos^2 x - \sin^2 x = 2\cos^2 x - 1 = 1 - 2\sin^2 x$ \\
 & $\tan 2x = \frac{2\tan x}{1 - \tan^2 x}$ \\
\midrule
Half Angle & $\sin^2 \frac{x}{2} = \frac{1 - \cos x}{2}$ \\
 & $\cos^2 \frac{x}{2} = \frac{1 + \cos x}{2}$ \\
 & $\tan^2 \frac{x}{2} = \frac{1 - \cos x}{1 + \cos x}$ \\
\bottomrule
\end{tabular}
\caption{Advanced Trigonometric Identities}
\label{tab:advanced_trig_identities}
\end{table}

\begin{table}[h]
\centering
\begin{tabular}{@{}ll@{}}
\toprule
Identity & Formula \\
\midrule
Product-to-Sum & $\sin A \cos B = \frac{1}{2}[\sin(A + B) + \sin(A - B)]$ \\
 & $\cos A \cos B = \frac{1}{2}[\cos(A + B) + \cos(A - B)]$ \\
 & $\sin A \sin B = \frac{1}{2}[\cos(A - B) - \cos(A + B)]$ \\
\midrule
Sum-to-Product & $\sin A + \sin B = 2 \sin(\frac{A + B}{2}) \cos(\frac{A - B}{2})$ \\
 & $\cos A + \cos B = 2 \cos(\frac{A + B}{2}) \cos(\frac{A - B}{2})$ \\
 & $\sin A - \sin B = 2 \cos(\frac{A + B}{2}) \sin(\frac{A - B}{2})$ \\
\bottomrule
\end{tabular}
\caption{Product-to-Sum and Sum-to-Product Identities}
\label{tab:product_sum_identities}
\end{table}

Below is what we call trigonometric identities hexagon.
\begin{figure}[h]
    \centering
    \includegraphics[width=0.5\linewidth]{Images/trighex.png}
    \caption{Trigonometric Identities Hexagon}
    \label{fig:enter-label}
\end{figure}


   

The trigonometric identities hexagon is a visual tool that allows you to quickly identify the relationships between the six main trigonometric functions: \(\sin(\theta)\), \(\cos(\theta)\), \(\tan(\theta)\), \(\cot(\theta)\), \(\sec(\theta)\), and \(\csc(\theta)\). Here’s how you can use this hexagon:

\begin{itemize}
    \item \textbf{Reciprocal Identities:} The edges of the hexagon connect pairs of functions that are reciprocals of each other. For example, \(\sin(\theta)\) is connected to \(\csc(\theta)\) because \(\csc(\theta) = \frac{1}{\sin(\theta)}\). Similarly, \(\cos(\theta)\) is connected to \(\sec(\theta)\), and \(\tan(\theta)\) is connected to \(\cot(\theta)\).
    
    \item \textbf{Quotient Identities:} The diagonals of the hexagon represent the quotient identities. For instance, the diagonal connecting \(\sin(\theta)\) and \(\cos(\theta)\) corresponds to the identity \(\tan(\theta) = \frac{\sin(\theta)}{\cos(\theta)}\). Similarly, the diagonal between \(\tan(\theta)\) and \(\sec(\theta)\) reflects the identity \(\tan(\theta) = \frac{\sin(\theta)}{\cos(\theta)}\) when combined with the reciprocal identity \(\sec(\theta) = \frac{1}{\cos(\theta)}\).
    
    \item \textbf{Pythagorean Identities:} The hexagon also encodes the Pythagorean identities, which can be derived by combining the relationships around the hexagon. For example, starting from the identity \(\sin^2(\theta) + \cos^2(\theta) = 1\), you can use the quotient identity \(\tan(\theta) = \frac{\sin(\theta)}{\cos(\theta)}\) to derive \(1 + \tan^2(\theta) = \sec^2(\theta)\). A similar process can be applied using \(\cot(\theta)\) and \(\csc(\theta)\) to derive the identity \(1 + \cot^2(\theta) = \csc^2(\theta)\).
\end{itemize}

By using the hexagon, you can easily recall or derive fundamental trigonometric identities without having to memorize each one individually. The geometric layout of the hexagon reveals the inherent symmetries and relationships between the functions, making it a valuable tool for both learning and applying trigonometry.


\section{Integration}
Below is the table of commonly seen integration.

\begin{table}[h]
\centering
\caption{Comprehensive Integration Table}
\begin{tabular}{ll ll}
\toprule
\text{Function } \(f(x)\) & \(\int f(x)\,dx\) & \text{Function } \(f(x)\) & \(\int f(x)\,dx\) \\
\midrule
\(x^n \ (n \neq -1)\) & \(\frac{x^{n+1}}{n+1} + C\) & \(\frac{1}{1+x^2}\) & \(\arctan x + C\) \\
\(\frac{1}{x}\) & \(\ln|x| + C\) & \(\frac{1}{\sqrt{1-x^2}}\) & \(\arcsin x + C\) \\
\(e^x\) & \(e^x + C\) & \(\frac{1}{\sqrt{x^2+a^2}}\) & \(\ln\left(x + \sqrt{x^2+a^2}\right) + C\) \\
\(a^x\) & \(\frac{a^x}{\ln a} + C\) & \(\frac{1}{\sqrt{x^2-a^2}}\) & \(\ln\left|x + \sqrt{x^2-a^2}\right| + C\) \\
\(\ln x\) & \(x\ln x - x + C\) & \(\frac{1}{a^2-x^2}\) & \(\frac{1}{2a}\ln\left|\frac{a+x}{a-x}\right| + C\) \\
\(\sin x\) & \(-\cos x + C\) & \(\arcsin x\) & \(x\arcsin x + \sqrt{1-x^2} + C\) \\
\(\cos x\) & \(\sin x + C\) & \(\arccos x\) & \(x\arccos x - \sqrt{1-x^2} + C\) \\
\(\tan x\) & \(-\ln|\cos x| + C\) & \(\arctan x\) & \(x\arctan x - \frac{1}{2}\ln(1+x^2) + C\) \\
\(\cot x\) & \(\ln|\sin x| + C\) & \(\text{arccot } x\) & \(x\text{ arccot} x + \frac{1}{2}\ln(1+x^2) + C\) \\
\(\sec x\) & \(\ln|\sec x + \tan x| + C\) & \(\text{arcsec } x\) & \(x\text{ arcsec} x - \ln|x + \sqrt{x^2-1}| + C\) \\
\(\csc x\) & \(-\ln|\csc x + \cot x| + C\) & \(\text{arccsc } x\) & \(x\text{ arccsc} x + \ln|x + \sqrt{x^2-1}| + C\) \\
\(\sec^2 x\) & \(\tan x + C\) & \(\sinh x\) & \(\cosh x + C\) \\
\(\csc^2 x\) & \(-\cot x + C\) & \(\cosh x\) & \(\sinh x + C\) \\
\(\sec x \tan x\) & \(\sec x + C\) & \(\tanh x\) & \(\ln(\cosh x) + C\) \\
\(\csc x \cot x\) & \(-\csc x + C\) & \(\coth x\) & \(\ln|\sinh x| + C\) \\
\bottomrule
\end{tabular}
\end{table}



\subsection{Integration Techniques}

\subsubsection{Integration by Parts}

\begin{definition}
Integration by parts is a technique based on the product rule of differentiation. It is useful when integrating the product of two functions.


$$
\int u \frac{dv}{dx} dx = uv - \int v \frac{du}{dx} dx
$$
Or alternatively:

$$\int u d v=u v-\int v d u$$

\end{definition}

\begin{theorem}[Proof of Integration by Parts]
Let $u = u(x)$ and $v = v(x)$. By the product rule:

$$
\frac{d}{dx}(uv) = u\frac{dv}{dx} + v\frac{du}{dx}
$$

Integrating both sides:

$$
\int \frac{d}{dx}(uv) dx = \int u\frac{dv}{dx} dx + \int v\frac{du}{dx} dx
$$

The left side simplifies to $uv$, and rearranging gives us the integration by parts formula.
\end{theorem}

\subsubsection{Integration by Substitution}

\begin{definition}
Integration by substitution is based on the chain rule of differentiation.

Formula:
If $u = g(x)$, then:

$$
\int f(g(x))g'(x)dx = \int f(u)du
$$
\end{definition}

\begin{theorem}[Proof of Integration by Substitution]
Let $F$ be an antiderivative of $f$. Then:

$$
\frac{d}{dx}[F(g(x))] = F'(g(x))g'(x) = f(g(x))g'(x)
$$

Integrating both sides:

$$
\int \frac{d}{dx}[F(g(x))] dx = \int f(g(x))g'(x)dx
$$

The left side simplifies to $F(g(x))$, which is equivalent to $\int f(u)du$ when we substitute $u = g(x)$.
\end{theorem}


\subsubsection{Trigonometric and Hyperbolic Function Substitution}

\begin{definition}
Trigonometric and hyperbolic function substitutions are specific techniques used for integrals involving expressions such as $\sqrt{a^2 - x^2}$, $\sqrt{a^2 + x^2}$, or $\sqrt{x^2 - a^2}$. These substitutions simplify the integral by transforming it into a form that is often easier to evaluate, depending on key identities associated with trigonometric or hyperbolic functions.

\begin{itemize}
    \item For $\sqrt{a^2 - x^2}$:
    \begin{itemize}
        \item Use $x = a\sin\theta$, where $\theta$ is a trigonometric angle. This substitution relies on the Pythagorean identity $\sin^2\theta + \cos^2\theta = 1$.
        \item Alternatively, use $x = a\tanh u$, where $u$ is a hyperbolic angle. This substitution depends on the hyperbolic identity $\cosh^2 u - \sinh^2 u = 1$.
    \end{itemize}
    
    \item For $\sqrt{a^2 + x^2}$:
    \begin{itemize}
        \item Use $x = a\tan\theta$, where $\theta$ is a trigonometric angle. This substitution is based on the identity $1 + \tan^2\theta = \sec^2\theta$.
        \item Alternatively, use $x = a\sinh u$, where $u$ is a hyperbolic angle. This substitution relies on the identity $\cosh^2 u - \sinh^2 u = 1$, specifically using the rearrangement $\cosh^2 u = 1 + \sinh^2 u$.
    \end{itemize}
    
    \item For $\sqrt{x^2 - a^2}$:
    \begin{itemize}
        \item Use $x = a\sec\theta$, where $\theta$ is a trigonometric angle. This substitution depends on the identity $\sec^2\theta - \tan^2\theta = 1$.
        \item Alternatively, use $x = a\cosh u$, where $u$ is a hyperbolic angle. This substitution is grounded in the identity $\cosh^2 u - \sinh^2 u = 1$.
    \end{itemize}
\end{itemize}
\end{definition}


\begin{example}[Substitution for $\sqrt{a^2 - x^2}$]
Using trigonometric substitution, let $x = a\sin\theta$. Then $dx = a\cos\theta \, d\theta$ and $\sqrt{a^2 - x^2} = a\cos\theta$.


Substituting $x = a\sin\theta$ into $\sqrt{a^2 - x^2}$:

$$
\sqrt{a^2 - (a\sin\theta)^2} = \sqrt{a^2(1 - \sin^2\theta)} = a\sqrt{\cos^2\theta} = a\cos\theta
$$

Alternatively, using hyperbolic substitution, let $x = a\tanh u$. Then $dx = a\text{sech}^2 u \, du$ and $\sqrt{a^2 - x^2} = \frac{a}{\cosh u}$.


Substituting $x = a\tanh u$ into $\sqrt{a^2 - x^2}$:

$$
\sqrt{a^2 - (a\tanh u)^2} = \sqrt{a^2\left(1 - \tanh^2 u\right)} = a\sqrt{\text{sech}^2 u} = \frac{a}{\cosh u}
$$
\end{example}

\begin{example}[Substitution for $\sqrt{a^2 + x^2}$]
Using trigonometric substitution, let $x = a\tan\theta$. Then $dx = a\sec^2\theta \, d\theta$ and $\sqrt{a^2 + x^2} = a\sec\theta$.

Proof:
Substituting $x = a\tan\theta$ into $\sqrt{a^2 + x^2}$:

$$
\sqrt{a^2 + (a\tan\theta)^2} = \sqrt{a^2(1 + \tan^2\theta)} = a\sqrt{\sec^2\theta} = a\sec\theta
$$

Alternatively, using hyperbolic substitution, let $x = a\sinh u$. Then $dx = a\cosh u \, du$ and $\sqrt{a^2 + x^2} = a\cosh u$.

Proof:
Substituting $x = a\sinh u$ into $\sqrt{a^2 + x^2}$:

$$
\sqrt{(a\sinh u)^2 + a^2} = \sqrt{a^2\sinh^2 u + a^2} = a\sqrt{\sinh^2 u + 1} = a\cosh u
$$
\end{example}

\begin{example}[Substitution for $\sqrt{x^2 - a^2}$]
Using trigonometric substitution, let $x = a\sec\theta$. Then $dx = a\sec\theta\tan\theta \, d\theta$ and $\sqrt{x^2 - a^2} = a\tan\theta$.

Proof:
Substituting $x = a\sec\theta$ into $\sqrt{x^2 - a^2}$:

$$
\sqrt{(a\sec\theta)^2 - a^2} = \sqrt{a^2\sec^2\theta - a^2} = a\sqrt{\sec^2\theta - 1} = a\tan\theta
$$

Alternatively, using hyperbolic substitution, let $x = a\cosh u$. Then $dx = a\sinh u \, du$ and $\sqrt{x^2 - a^2} = a\sinh u$.

Proof:
Substituting $x = a\cosh u$ into $\sqrt{x^2 - a^2}$:

$$
\sqrt{(a\cosh u)^2 - a^2} = \sqrt{a^2\cosh^2 u - a^2} = a\sqrt{\cosh^2 u - 1} = a\sinh u
$$
\end{example}

\subsubsection{Important Hyperbolic Function Identities}
Here are some important properties of hyperbolic functions. Most of them are quite similar, or even identical to trigonometric functions, however, do note that there are a lot of nuances.
\begin{table}[h]
\centering
\begin{tabular}{@{}ll@{}}
\toprule
Identity & Formula \\
\midrule
Pythagorean & $\cosh^2 x - \sinh^2 x = 1$ \\
 & $\tanh^2 x + \text{sech}^2 x = 1$ \\
 & $\coth^2 x - \text{csch}^2 x = 1$ \\
\midrule
Reciprocal & $\text{sech} \, x = \frac{1}{\cosh x}$, $\text{csch} \, x = \frac{1}{\sinh x}$ \\
 & $\coth x = \frac{1}{\tanh x}$ \\
\midrule
Quotient & $\tanh x = \frac{\sinh x}{\cosh x}$, $\coth x = \frac{\cosh x}{\sinh x}$ \\
\midrule
Even-Odd & $\sinh(-x) = -\sinh x$, $\cosh(-x) = \cosh x$ \\
 & $\tanh(-x) = -\tanh x$, $\coth(-x) = -\coth x$ \\
\midrule
Periodicity & $\sinh(x + 2\pi i) = \sinh x$, $\cosh(x + 2\pi i) = \cosh x$ \\
 & $\tanh(x + \pi i) = \tanh x$ \\
\bottomrule
\end{tabular}
\caption{Basic Hyperbolic Function Identities}
\label{tab:basic_hyperbolic_identities}
\end{table}

\begin{table}[h]
\centering
\begin{tabular}{@{}ll@{}}
\toprule
Identity & Formula \\
\midrule
Sum & $\sinh(A + B) = \sinh A \cosh B + \cosh A \sinh B$ \\
 & $\cosh(A + B) = \cosh A \cosh B + \sinh A \sinh B$ \\
 & $\tanh(A + B) = \frac{\tanh A + \tanh B}{1 + \tanh A \tanh B}$ \\
\midrule
Difference & $\sinh(A - B) = \sinh A \cosh B - \cosh A \sinh B$ \\
 & $\cosh(A - B) = \cosh A \cosh B - \sinh A \sinh B$ \\
 & $\tanh(A - B) = \frac{\tanh A - \tanh B}{1 - \tanh A \tanh B}$ \\
\midrule
Double Angle & $\sinh 2x = 2\sinh x \cosh x$ \\
 & $\cosh 2x = \cosh^2 x + \sinh^2 x = 2\cosh^2 x - 1 = 1 + 2\sinh^2 x$ \\
 & $\tanh 2x = \frac{2\tanh x}{1 + \tanh^2 x}$ \\
\midrule
Half Angle & $\sinh^2 \frac{x}{2} = \frac{\cosh x - 1}{2}$ \\
 & $\cosh^2 \frac{x}{2} = \frac{\cosh x + 1}{2}$ \\
 & $\tanh^2 \frac{x}{2} = \frac{\cosh x - 1}{\cosh x + 1}$ \\
\bottomrule
\end{tabular}
\caption{Advanced Hyperbolic Function Identities}
\label{tab:advanced_hyperbolic_identities}
\end{table}

\begin{table}[h]
\centering
\begin{tabular}{@{}ll@{}}
\toprule
Identity & Formula \\
\midrule
Product-to-Sum & $\sinh A \cosh B = \frac{1}{2}[\sinh(A + B) + \sinh(A - B)]$ \\
 & $\cosh A \cosh B = \frac{1}{2}[\cosh(A + B) + \cosh(A - B)]$ \\
 & $\sinh A \sinh B = \frac{1}{2}[\cosh(A + B) - \cosh(A - B)]$ \\
\midrule
Sum-to-Product & $\sinh A + \sinh B = 2 \sinh(\frac{A + B}{2}) \cosh(\frac{A - B}{2})$ \\
 & $\cosh A + \cosh B = 2 \cosh(\frac{A + B}{2}) \cosh(\frac{A - B}{2})$ \\
 & $\sinh A - \sinh B = 2 \cosh(\frac{A + B}{2}) \sinh(\frac{A - B}{2})$ \\
\bottomrule
\end{tabular}
\caption{Product-to-Sum and Sum-to-Product Hyperbolic Identities}
\label{tab:product_sum_hyperbolic_identities}
\end{table}

\part{Linear Algebra}
\chapter{Geometry in Euclidean Space}
\section{Vector and Linear Combination}

\section{Point, Line, Plane, and Beyond}

\section{Parametric Curves and Surfaces}

\chapter{Matrices and System of Linear Equations}

\chapter{Vector Space and Subspace}

\section{Definitions and Properties of Vector Space}
\subsection{$\mathbb{C}$}
\begin{exercise}
Show that for every $\alpha \in \mathbb{C}$, there exists a unique $\beta \in \mathbf{C}$ such that $\alpha + \beta = 0$.
\end{exercise}

\begin{solution}
Given any $\alpha \in \mathbb{C}$, we need to find a $\beta \in \mathbf{C}$ such that $\alpha + \beta = 0$. Let $\beta = -\alpha$. Then:
\[
\alpha + \beta = \alpha + (-\alpha) = 0.
\]
Thus, for every $\alpha \in \mathbb{C}$, $\beta = -\alpha$ satisfies the equation $\alpha + \beta = 0$. To show uniqueness, assume there exists another $\beta' \in \mathbf{C}$ such that $\alpha + \beta' = 0$. Then:
\[
\alpha + \beta = \alpha + \beta' = 0 \implies \beta = \beta'.
\]
Therefore, $\beta = -\alpha$ is the unique solution.
\end{solution}

\begin{exercise}
Show that for every $\alpha \in \mathbb{C}$ with $\alpha \neq 0$, there exists a unique $\beta \in \mathbb{C}$ such that $\alpha \beta = 1$.
\end{exercise}

\begin{solution}
Given any $\alpha \in \mathbf{C}$ with $\alpha \neq 0$, we need to find a $\beta \in \mathbb{C}$ such that $\alpha \beta = 1$. Let $\beta = \frac{1}{\alpha}$. Then:
\[
\alpha \beta = \alpha \left( \frac{1}{\alpha} \right) = 1.
\]
Thus, for every $\alpha \in \mathbb{C}$ with $\alpha \neq 0$, $\beta = \frac{1}{\alpha}$ satisfies the equation $\alpha \beta = 1$. To show uniqueness, assume there exists another $\beta' \in \mathbb{C}$ such that $\alpha \beta' = 1$. Then:
\[
\alpha \beta = \alpha \beta' = 1 \implies \beta = \beta'.
\]
Therefore, $\beta = \frac{1}{\alpha}$ is the unique solution.
\end{solution}

\begin{exercise}
Show that $(ab)x = a(bx)$ for all $x \in \mathbf{F}^{n}$ and all $a, b \in \mathbf{F}$.
\end{exercise}

\begin{solution}
Let $x \in \mathbf{F}^{n}$ and $a, b \in \mathbf{F}$. We want to show that $(ab)x = a(bx)$. 

First, recall that in a vector space over a field $\mathbf{F}$, the scalar multiplication is associative. This means that for any scalar $c \in \mathbf{F}$ and vector $y \in \mathbf{F}^{n}$, we have $c(y) = (cy)$. 

Now, consider the left-hand side of the equation:
\[
(ab)x
\]
By the definition of scalar multiplication in a vector space, multiplying a vector by a scalar is the same as multiplying each component of the vector by the scalar. Hence, we can write:
\[
(ab)x = (ab) \cdot x
\]

Next, consider the right-hand side of the equation:
\[
a(bx)
\]
First, compute $bx$:
\[
bx = b \cdot x
\]
Then, multiplying the result by $a$ gives:
\[
a(bx) = a \cdot (b \cdot x)
\]

By the associativity of scalar multiplication, we have:
\[
a \cdot (b \cdot x) = (a \cdot b) \cdot x = (ab)x
\]

Thus, we have shown that:
\[
(ab)x = a(bx)
\]
for all $x \in \mathbf{F}^{n}$ and all $a, b \in \mathbf{F}$.
\end{solution}

\begin{exercise}
Show that $(a+b)x = ax + bx$ for all $a, b \in \mathbf{F}$ and all $x \in \mathbf{F}^{n}$.
\end{exercise}

\begin{solution}
Let $x \in \mathbf{F}^{n}$ and $a, b \in \mathbf{F}$. We want to show that $(a+b)x = ax + bx$.

Recall that in a vector space over a field $\mathbf{F}$, scalar multiplication distributes over vector addition. This means that for any scalars $c, d \in \mathbf{F}$ and any vector $y \in \mathbf{F}^{n}$, we have:
\[
(c + d)y = cy + dy.
\]

Now, consider the left-hand side of the equation:
\[
(a + b)x.
\]
By the definition of scalar multiplication in a vector space, we have:
\[
(a + b)x = (a + b) \cdot x.
\]

Next, consider the right-hand side of the equation:
\[
ax + bx.
\]
First, compute $ax$ and $bx$:
\[
ax = a \cdot x,
\]
\[
bx = b \cdot x.
\]
Then, add these results together:
\[
ax + bx = a \cdot x + b \cdot x.
\]

By the distributive property of scalar multiplication over vector addition, we have:
\[
(a + b) \cdot x = a \cdot x + b \cdot x.
\]

Thus, we have shown that:
\[
(a + b)x = ax + bx
\]
for all $a, b \in \mathbf{F}$ and all $x \in \mathbf{F}^{n}$.
\end{solution}

\begin{exercise}
Show that $\lambda(x + y) = \lambda x + \lambda y$ for all $\lambda \in \mathbf{F}$ and all $x, y \in \mathbf{F}^{n}$.
\end{exercise}

\begin{solution}
Let $x, y \in \mathbf{F}^{n}$ and $\lambda \in \mathbf{F}$. We want to show that $\lambda(x + y) = \lambda x + \lambda y$.

Recall that in a vector space over a field $\mathbf{F}$, scalar multiplication distributes over vector addition. This means that for any scalar $\lambda \in \mathbf{F}$ and any vectors $x, y \in \mathbf{F}^{n}$, we have:
\[
\lambda(x + y) = \lambda x + \lambda y.
\]

Now, consider the left-hand side of the equation:
\[
\lambda(x + y).
\]
By the definition of vector addition and scalar multiplication in a vector space, we have:
\[
x + y \in \mathbf{F}^{n}.
\]
Then, scalar multiplication distributes over this vector addition:
\[
\lambda(x + y).
\]

Next, consider the right-hand side of the equation:
\[
\lambda x + \lambda y.
\]
First, compute $\lambda x$ and $\lambda y$:
\[
\lambda x \in \mathbf{F}^{n},
\]
\[
\lambda y \in \mathbf{F}^{n}.
\]
Then, add these results together:
\[
\lambda x + \lambda y \in \mathbf{F}^{n}.
\]

By the distributive property of scalar multiplication over vector addition, we have:
\[
\lambda(x + y) = \lambda x + \lambda y.
\]

Thus, we have shown that:
\[
\lambda(x + y) = \lambda x + \lambda y
\]
for all $\lambda \in \mathbf{F}$ and all $x, y \in \mathbf{F}^{n}$.
\end{solution}

\section{$F^n$ and $F^S$}
\begin{definition}[$F^n$]
Let $\mathbf{F}$ be a field and $n$ be a positive integer. The set $\mathbf{F}^n$ is defined as the set of all $n$-tuples of elements from $\mathbf{F}$:
\[
\mathbf{F}^n = \{ (x_1, x_2, \ldots, x_n) \mid x_i \in \mathbf{F} \text{ for } i = 1, 2, \ldots, n \}.
\]
Elements of $\mathbf{F}^n$ are called vectors, and $\mathbf{F}^n$ is called an $n$-dimensional vector space over the field $\mathbf{F}$.
\end{definition}
\begin{definition}[$F^S$]
Let $\mathbf{F}$ be a field and $S$ be a set. The set $\mathbf{F}^S$ is defined as the set of all functions from $S$ to $\mathbf{F}$:
\[
\mathbf{F}^S = \{ f \mid f: S \to \mathbf{F} \}.
\]
Each function $f \in \mathbf{F}^S$ assigns an element of $\mathbf{F}$ to each element of $S$. $\mathbf{F}^S$ is called a vector space over the field $\mathbf{F}$.
\end{definition}
\begin{exercise}
Suppose $a \in \mathbf{F}, v \in V$, and $av = 0$. Prove that $a = 0$ or $v = 0$. 
\end{exercise}

\begin{solution}
We will prove this statement by contradiction.

Assume the contrary, that is, suppose $a \neq 0$ and $v \neq 0$. 

Since $a \neq 0$, it has a multiplicative inverse $a^{-1}$ in the field $\mathbf{F}$. 

Now, consider the vector $a^{-1}(av)$. Using the associativity of scalar multiplication, we have:
\[
a^{-1}(av) = (a^{-1}a)v = 1v = v.
\]

Given that $av = 0$, we substitute this into the above expression:
\[
a^{-1}(0) = 0.
\]

So, we have:
\[
v = 0,
\]
which contradicts our assumption that $v \neq 0$.

Therefore, our assumption that both $a \neq 0$ and $v \neq 0$ must be false. Hence, it must be that either $a = 0$ or $v = 0$.
\end{solution}

\begin{exercise}
Prove that $-(-v) = v$ for every $v \in V$.
\end{exercise}

\begin{solution}
By definition of the additive inverse, we have:
\[
(-v) + (-(-v)) = 0 \quad \text{and} \quad v + (-v) = 0.
\]

This implies that both $v$ and $-(-v)$ are additive inverses of $-v$. By the uniqueness of the additive inverse, it follows that:
\[
-(-v) = v.
\]
\end{solution}

\begin{exercise}
Suppose $v, w \in V$. Explain why there exists a unique $x \in V$ such that $v + 3x = w$.
\end{exercise}

\begin{solution}
To find a unique $x \in V$ such that $v + 3x = w$, we proceed as follows:

First, isolate $3x$ by subtracting $v$ from both sides of the equation:
\[
v + 3x = w \implies 3x = w - v.
\]

Since $3 \neq 0$ in the field $\mathbf{F}$, there exists a multiplicative inverse of $3$, denoted as $\frac{1}{3}$. Multiply both sides of the equation $3x = w - v$ by $\frac{1}{3}$:
\[
x = \frac{1}{3} (w - v).
\]

We have now found an $x \in V$ that satisfies the equation. To prove the uniqueness of $x$, suppose there exists another $x' \in V$ such that $v + 3x' = w$. Then:
\[
3x' = w - v.
\]

Multiplying both sides by $\frac{1}{3}$, we obtain:
\[
x' = \frac{1}{3} (w - v).
\]

Thus, $x = x'$, showing that the solution $x$ is unique.

Therefore, for any $v, w \in V$, there exists a unique $x \in V$ such that $v + 3x = w$.
\end{solution}

\begin{exercise}
Show that in the definition of a vector space, the additive inverse condition can be replaced with the condition that
\[
0 v = 0 \quad \text{for all } v \in V,
\]
where the 0 on the left side is the scalar zero, and the 0 on the right side is the additive identity of $V$.
\end{exercise}

\begin{solution}
We begin by noting the properties of scalar multiplication and the existence of the zero vector in a vector space. The condition $0v = 0_V$ for all $v \in V$ implies that multiplying any vector by the scalar zero yields the zero vector, consistent with vector space axioms.

To demonstrate the additive inverse property using $0v = 0_V$, consider:
\[
0v = (0+0)v = 0v + 0v.
\]
Subtracting $0v$ from both sides, we have:
\[
0v = 0_V.
\]

We leverage the distributive property of scalar multiplication over vector addition to find an additive inverse:
\[
1v + (-1)v = (1-1)v = 0v = 0_V.
\]
This simplifies to:
\[
v + (-1)v = 0_V,
\]
where $(-1)v$ acts as the additive inverse of $v$. Thus, $(-1)v = -v$, and we confirm:
\[
v + (-v) = 0_V.
\]

Thus, the condition $0v = 0_V$ indeed implies the existence of an additive inverse for every vector $v$ in $V$, allowing us to replace the explicit requirement for an additive inverse with the condition $0v = 0_V$ in the definition of a vector space.
\end{solution}

\begin{exercise}
Let $\infty$ and $-\infty$ denote two distinct objects, neither of which is in $\mathbf{R}$. Define an addition and scalar multiplication on $\mathbf{R} \cup\{\infty,-\infty\}$ as you could guess from the notation. Specifically, the sum and product of two real numbers is as usual, and for $t \in \mathbf{R}$ define

\[
t \infty=\left\{\begin{array}{ll}
-\infty & \text { if } t<0 \\
0 & \text { if } t=0, \\
\infty & \text { if } t>0
\end{array} \quad t(-\infty)= \begin{cases}\infty & \text { if } t<0 \\
0 & \text { if } t=0 \\
-\infty & \text { if } t>0\end{cases}\right.
\]

and

\[
\begin{aligned}
t+\infty & =\infty+t=\infty+\infty=\infty \\
t+(-\infty) & =(-\infty)+t=(-\infty)+(-\infty)=-\infty \\
\infty+(-\infty) & =(-\infty)+\infty=0
\end{aligned}
\]

With these operations of addition and scalar multiplication, is $\mathbf{R} \cup\{\infty,-\infty\}$ a vector space over $\mathbf{R}$ ? Explain.
\end{exercise}
\begin{solution}
This is not a vector space over $\mathbb{R}$. Consider the distributive properties. If this is a vector space over $\mathbb{R}$, we will have
$$
\infty=(2+(-1)) \infty=2 \infty+(-1) \infty=\infty+(-\infty)=0 \text {. }
$$

Hence for any $t \in \mathbb{R}$, one has
$$
t=0+t=\infty+t=\infty=0 .
$$

We get a contradiction since zero vector is unique.
\end{solution}

\begin{exercise}
Suppose $V$ is a real vector space.
The complexification of $V$, denoted by $V_{\mathrm{C}}$, equals $V \times V$. An element of $V_{\mathrm{C}}$ is an ordered pair $(u, v)$, where $u, v \in V$, but we write this as $u+i v$.
Addition on $V_{\mathrm{C}}$ is defined by

\[
\left(u_{1}+i v_{1}\right)+\left(u_{2}+i v_{2}\right)=\left(u_{1}+u_{2}\right)+i\left(v_{1}+v_{2}\right)
\]

for all $u_{1}, v_{1}, u_{2}, v_{2} \in V$.
- Complex scalar multiplication on $V_{\mathrm{C}}$ is defined by

\[
(a+b i)(u+i v)=(a u-b v)+i(a v+b u)
\]

for all $a, b \in \mathbf{R}$ and all $u, v \in V$.

Prove that with the definitions of addition and scalar multiplication as above, $V_{\mathrm{C}}$ is a complex vector space.

Think of $V$ as a subset of $V_{\mathrm{C}}$ by identifying $u \in V$ with $u+i 0$. The construction of $V_{\mathrm{C}}$ from $V$ can then be thought of as generalizing the construction of $\mathbf{C}^{n}$ from $\mathbf{R}^{n}$.
\end{exercise}

\begin{solution}
To prove that \( V_C \) is closed under complex scalar multiplication, we need to show that for any complex scalar \( a + bi \in \mathbb{C} \) and any element \( u + iv \in V_C \), the product \( (a + bi)(u + iv) \) is also in \( V_C \).

Recall the definition of complex scalar multiplication on \( V_C \):
\[ (a + bi)(u + iv) = (au - bv) + i(av + bu) \]

We need to verify that the result of this multiplication, \( (au - bv) + i(av + bu) \), is an element of \( V_C \).

\begin{itemize}
    \item The real part of the product is \( au - bv \).
    \item The imaginary part of the product is \( av + bu \).
\end{itemize}

Since \( u, v \in V \) and \( V \) is a real vector space, we know the following:
\begin{itemize}
    \item \( u, v \in V \implies au, bv \in V \) because \( V \) is closed under scalar multiplication.
    \item \( V \) is closed under addition, so \( au - bv \in V \).
    \item Similarly, \( av, bu \in V \) and \( V \) is closed under addition, so \( av + bu \in V \).
\end{itemize}

Therefore, both \( au - bv \in V \) and \( av + bu \in V \), implying that \( (au - bv) + i(av + bu) \in V_C \).

Thus, we have shown that \( (a + bi)(u + iv) \in V_C \), confirming that \( V_C \) is closed under complex scalar multiplication.

\end{solution}


\section{Subspace of Vector Space}
\begin{definition}[Subspace]
    A subset $U$ of $V$ is called a subspace of $V$ if $U$ is also a vector space with the same additive identity, addition, and scalar multiplication as on $V$.
\end{definition}
\subsection{Criteria of Subspace}
\begin{proposition}[Conditions for a Subspace]
    A subset $U$ of $V$ is a subspace of $V$ if and only if $U$ satisfies the following three conditions.

\begin{enumerate}
    \item \textbf{additive identity}: $0 \in U$.
    \item \textbf{closed under addition}: $u, w \in U$ implies $u+w \in U$.
    \item \textbf{closed under scalar multiplication}: $a \in \mathbb{F}$ and $u \in U$ implies $a u \in U$.
\end{enumerate}

\end{proposition}

\begin{exercise}
1 For each of the following subsets of $\mathbb{F}^{3}$, determine whether it is a subspace of $\mathbb{F}^{3}$.

(a) $\left\{\left(x_{1}, x_{2}, x_{3}\right) \in \mathbb{F}^{3}: x_{1}+2 x_{2}+3 x_{3}=0\right\}$

(b) $\left\{\left(x_{1}, x_{2}, x_{3}\right) \in \mathbb{F}^{3}: x_{1}+2 x_{2}+3 x_{3}=4\right\}$

(c) $\left\{\left(x_{1}, x_{2}, x_{3}\right) \in \mathbb{F}^{3}: x_{1} x_{2} x_{3}=0\right\}$

(d) $\left\{\left(x_{1}, x_{2}, x_{3}\right) \in \mathbb{F}^{3}: x_{1}=5 x_{3}\right\}$
\end{exercise}
\begin{solution}
We apply the rules given.
\begin{enumerate}
    \item[(a)] \(\left\{ \left( x_{1}, x_{2}, x_{3} \right) \mid x_{1} + 2x_{2} + 3x_{3} = 0 \right\}\)
    \begin{enumerate}
        \item \textbf{Additive Identity:} \(0 + 2 \cdot 0 + 3 \cdot 0 = 0\), so \( \mathbf{0} \in U \).
        \item \textbf{Closed under Addition:} If \(u = (u_{1}, u_{2}, u_{3}) \in U\) and \(w = (w_{1}, w_{2}, w_{3}) \in U\), then \((u_{1} + w_{1}) + 2(u_{2} + w_{2}) + 3(u_{3} + w_{3}) = 0\), so \(u + w \in U\).
        \item \textbf{Closed under Scalar Multiplication:} If \(u = (u_{1}, u_{2}, u_{3}) \in U\) and \(a \in \mathbb{F}\), then \(a(u_{1} + 2u_{2} + 3u_{3}) = 0\), so \(au \in U\).
    \end{enumerate}
    Hence, \( U \) is a subspace of \(\mathbb{F}^{3}\).

    \item[(b)] \(\left\{ \left( x_{1}, x_{2}, x_{3} \right) \mid x_{1} + 2x_{2} + 3x_{3} = 4 \right\}\)
    \begin{enumerate}
        \item \textbf{Additive Identity:} \(0 + 2 \cdot 0 + 3 \cdot 0 = 0 \neq 4\), so \( \mathbf{0} \notin U \).
    \end{enumerate}
    Hence, \( U \) is not a subspace of \(\mathbb{F}^{3}\).

    \item[(c)] \(\left\{ \left( x_{1}, x_{2}, x_{3} \right) \mid x_{1} x_{2} x_{3} = 0 \right\}\)
    \begin{enumerate}
        \item \textbf{Additive Identity:} \(0 \cdot 0 \cdot 0 = 0\), so \( \mathbf{0} \in U \).
        \item \textbf{Closed under Addition:} \(u = (1, 0, 0) \in U\) and \(w = (0, 1, 0) \in U\), but \(u + w = (1, 1, 0) \notin U\).
    \end{enumerate}
    Hence, \( U \) is not a subspace of \(\mathbb{F}^{3}\).

    \item[(d)] \(\left\{ \left( x_{1}, x_{2}, x_{3} \right) \mid x_{1} = 5x_{3} \right\}\)
    \begin{enumerate}
        \item \textbf{Additive Identity:} \(0 = 5 \cdot 0\), so \( \mathbf{0} \in U \).
        \item \textbf{Closed under Addition:} If \(u = (5u_{3}, u_{2}, u_{3}) \in U\) and \(w = (5w_{3}, w_{2}, w_{3}) \in U\), then \(u + w = (5(u_{3} + w_{3}), u_{2} + w_{2}, u_{3} + w_{3}) \in U\).
        \item \textbf{Closed under Scalar Multiplication:} If \(u = (5u_{3}, u_{2}, u_{3}) \in U\) and \(a \in \mathbb{F}\), then \(au = (5(au_{3}), au_{2}, au_{3}) \in U\).
    \end{enumerate}
    Hence, \( U \) is a subspace of \(\mathbb{F}^{3}\).
\end{enumerate}
\end{solution}

\begin{exercise}
Justify each of the following sets is a subspace, providing necessary conditions where applicable.

\begin{enumerate}
    \item[(a)] If \( b \in \mathbb{F} \), then \(\left\{(x_{1}, x_{2}, x_{3}, x_{4}) \in \mathbb{F}^{4} \mid x_{3} = 5x_{4} + b \right\}\) is a subspace of \(\mathbb{F}^{4}\) if and only if \( b = 0 \).
    
    \item[(b)] The set of continuous real-valued functions on the interval \([0,1]\) is a subspace of \(\mathbb{R}^{[0,1]}\).
    
    \item[(c)] The set of differentiable real-valued functions on \(\mathbb{R}\) is a subspace of \(\mathbb{R}^{\mathbb{R}}\).
    
    \item[(d)] The set of differentiable real-valued functions \( f \) on the interval \((0,3)\) such that \( f'(2) = b \) is a subspace of \(\mathbb{R}^{(0,3)}\) if and only if \( b = 0 \).
    
    \item[(e)] The set of all sequences of complex numbers with limit \( 0 \) is a subspace of \(\mathbb{C}^{\infty}\).
\end{enumerate}
\end{exercise}
\begin{solution}
    By conditions of a subspace.
    \begin{enumerate}
        \item[(a)] Let \( x_1 = m \), \( x_2 = n \), \( x_4 = a \), \( x_3 = 5a + b \). Since \( b \in \mathbb{F} \) and \( x_1, x_2, x_4 \in \mathbb{F} \), \( x_3 \in \mathbb{F} \). Thus we can show that the subspace \( U \) is closed under multiplication and addition. Now we only need to show the existence of additive identity. We only need to show that \( x_3 \) can be 0, since the rest of the components are all in \( \mathbb{F} \). As \( b \in \mathbb{F} \), when \( b = 0 \) and \( x_4 = 0 \), \( x_3 = 0 \), and thus \( \mathbf{0} \in \mathbb{F}^4 \). Hence, \( U \) is a subspace of \( \mathbb{F}^4 \) when \( b = 0 \).
        
        \item[(b)] The set of continuous real-valued functions on the interval \([0,1]\), denoted by \( C([0,1]) \), is a subspace of \( \mathbb{R}^{[0,1]} \). It is obvious that the zero function \( f(x) = 0 \) for all \( x \in [0,1] \) is continuous and hence belongs to \( C([0,1]) \). Also, if \( f \) and \( g \) are continuous functions on \([0,1]\), then \( f + g \) and \( cf \) (for any scalar \( c \in \mathbb{R} \)) are also continuous on \([0,1]\). Therefore, \( C([0,1]) \) is closed under addition and scalar multiplication, making it a subspace.
        
        \item[(c)] The set of differentiable real-valued functions on \( \mathbb{R} \), denoted by \( D(\mathbb{R}) \), is a subspace of \( \mathbb{R}^{\mathbb{R}} \). The zero function \( f(x) = 0 \) for all \( x \in \mathbb{R} \) is differentiable. If \( f \) and \( g \) are differentiable functions, then their sum \( f + g \) and scalar multiple \( cf \) are also differentiable. Therefore, \( D(\mathbb{R}) \) is closed under addition and scalar multiplication, making it a subspace.
        
        \item[(d)] The set of differentiable real-valued functions \( f \) on the interval \( (0,3) \) such that \( f'(2) = b \), denoted by \( D_b((0,3)) \), is a subspace of \( \mathbb{R}^{(0,3)} \) if and only if \( b = 0 \). To be a subspace, it must include the zero function, which means \( f'(2) = 0 \) for the zero function. This implies \( b = 0 \). For \( b = 0 \), the set \( D_0((0,3)) \) includes the zero function. Additionally, if \( f \) and \( g \) are in \( D_0((0,3)) \), then \( (f + g)'(2) = f'(2) + g'(2) = 0 + 0 = 0 \), and if \( f \) is in \( D_0((0,3)) \) and \( c \) is a scalar, then \( (cf)'(2) = cf'(2) = c \cdot 0 = 0 \). Thus, \( D_0((0,3)) \) is closed under addition and scalar multiplication, making it a subspace.
        
        \item[(e)] The set of all sequences of complex numbers with limit \( 0 \) is a subspace of \( \mathbb{C}^{\infty} \). Let \( S \) be the set of all sequences of complex numbers with limit \( 0 \). The zero sequence \( (0, 0, 0, \ldots) \) is in \( S \). If \( (a_n) \) and \( (b_n) \) are in \( S \), then \( \lim_{n \to \infty} a_n = 0 \) and \( \lim_{n \to \infty} b_n = 0 \). Thus, \( \lim_{n \to \infty} (a_n + b_n) = \lim_{n \to \infty} a_n + \lim_{n \to \infty} b_n = 0 + 0 = 0 \), so \( (a_n + b_n) \in S \). If \( (a_n) \in S \) and \( c \in \mathbb{C} \), then \( \lim_{n \to \infty} (c a_n) = c \lim_{n \to \infty} a_n = c \cdot 0 = 0 \), so \( (c a_n) \in S \). Thus, \( S \) is closed under addition and scalar multiplication, making it a subspace.
    \end{enumerate}
\end{solution}

\begin{exercise}
    Show that the set of differentiable real-valued functions $f$ on the interval $(-4,4)$ such that $f^{\prime}(-1)=3 f(2)$ is a subspace of $\mathbf{R}^{(-4,4)}$.
\end{exercise}
\begin{solution}
To prove that the set \( V = \{ f : (-4,4) \to \mathbb{R} \mid f \text{ is differentiable and } f'(-1) = 3f(2) \} \) is a subspace of \( \mathbb{R}^{(-4,4)} \), we need to verify the following three conditions:

\begin{enumerate}
    \item \( V \) contains the zero vector (zero function).
    \item \( V \) is closed under addition.
    \item \( V \) is closed under scalar multiplication.
\end{enumerate}

\paragraph{1. Zero function:}

Consider the zero function \( f(x) = 0 \). We need to verify that it belongs to \( V \).

\[
f'(x) = 0 \text{ for all } x \in (-4,4)
\]

Specifically, \( f'(-1) = 0 \) and \( f(2) = 0 \). Therefore,

\[
f'(-1) = 0 = 3 \cdot 0 = 3f(2)
\]

Hence, the zero function belongs to \( V \).

\paragraph{2. Closed under addition:}

Assume \( f \) and \( g \) both belong to \( V \). We need to show that \( f + g \) also belongs to \( V \).

Since \( f \) and \( g \) belong to \( V \), we have

\[
f'(-1) = 3f(2) \quad \text{and} \quad g'(-1) = 3g(2)
\]

Consider \( h = f + g \). We need to check if \( h \) satisfies \( h'(-1) = 3h(2) \).

Since \( h = f + g \), we have

\[
h' = f' + g'
\]

Thus,

\[
h'(-1) = f'(-1) + g'(-1)
\]

and

\[
h(2) = f(2) + g(2)
\]

Therefore,

\[
h'(-1) = f'(-1) + g'(-1) = 3f(2) + 3g(2) = 3(f(2) + g(2)) = 3h(2)
\]

This shows that \( f + g \) also satisfies \( f'(-1) = 3f(2) \), so \( f + g \) belongs to \( V \).

\paragraph{3. Closed under scalar multiplication:}

Assume \( f \) belongs to \( V \) and \( c \) is a scalar. We need to show that \( cf \) also belongs to \( V \).

Since \( f \) belongs to \( V \), we have

\[
f'(-1) = 3f(2)
\]

Consider \( h = cf \). We need to check if \( h \) satisfies \( h'(-1) = 3h(2) \).

Since \( h = cf \), we have

\[
h' = c f'
\]

Thus,

\[
h'(-1) = c f'(-1)
\]

and

\[
h(2) = c f(2)
\]

Therefore,

\[
h'(-1) = c f'(-1) = c \cdot 3f(2) = 3c f(2) = 3h(2)
\]

This shows that \( cf \) also satisfies \( f'(-1) = 3f(2) \), so \( cf \) belongs to \( V \).
\end{solution}

\begin{exercise}
    Suppose $b \in \mathbf{R}$. Show that the set of continuous real-valued functions $f$ on the interval $[0,1]$ such that $\int_{0}^{1} f=b$ is a subspace of $\mathbf{R}^{[0,1]}$ if and only if $b=0$.
\end{exercise}

\begin{solution}
Denote the set of continuous real-valued functions $f$ on the interval $[0,1]$ such that $\int_0^1 f=b$ by $V_b$.

If $V_b$ is a subspace of $\mathbb{R}^{[0,1]}$, then for any $f \in V_b$, we have $\int_0^1 f=b$. Because $V_b$ is a subspace of $\mathbb{R}^n$, it follows that $k f \in V_b$ for any $k \in \mathbb{R}$. Hence
$$
b=\int_0^1(k f)=k \int_0^1 f=k b, \quad \text { for all } k \in \mathbb{R},
$$
this happens if and only if $b=0$.

Now if $b=0$, then for any $f, g \in V_0$ and $\lambda \in \mathbb{R}$. We have that
$$
\int_0^1(f+g)=\int_0^1 f+\int_0^1 g=0+0=0
$$
and $f+g$ is continuous real-valued functions since $f$ and $g$ are. This deduces $f+g \in V_0$, i.e. $V_0$ is closed under addition. Similarly,
$$
\int_0^1(\lambda f)=\lambda \int_0^1 f=k 0=0
$$
and $\lambda f$ is continuous real-valued functions since $f$ is. This implies $\lambda f \in V_0$, i.e. $V_0$ is closed under scalar multiplication. On the other hand, the constant function $f \equiv 0 \in V_0$, which is also the additive identity in $\mathbb{R}^{[0,1]}$. Hence $V_0$ is a subspace of $\mathbb{R}^n$ by 1.34 .
\end{solution}

\begin{exercise}
    A function $f: \mathbf{R} \rightarrow \mathbf{R}$ is called periodic if there exists a positive number $p$ such that $f(x)=f(x+p)$ for all $x \in \mathbf{R}$. Is the set of periodic functions from $\mathbf{R}$ to $\mathbf{R}$ a subspace of $\mathbf{R}^{\mathbf{R}}$ ? Explain.
\end{exercise}
\begin{solution}
Let's analyze whether the set of periodic functions from \(\mathbf{R}\) to \(\mathbf{R}\) forms a subspace of \(\mathbf{R}^{\mathbf{R}} \), the vector space of all functions from \(\mathbf{R}\) to \(\mathbf{R}\).

To be a subspace, the set of periodic functions must satisfy three conditions:

1. \textbf{Contain the zero vector:} The zero vector in \(\mathbf{R}^{\mathbf{R}}\) is the zero function \(f(x) = 0\) for all \(x \in \mathbf{R}\). This function is trivially periodic with any period \(p > 0\), so the zero function is in the set of periodic functions.

2. \textbf{Closed under addition:} Suppose \(f\) and \(g\) are periodic functions with periods \(p_f\) and \(p_g\), respectively. That means \(f(x) = f(x + p_f)\) and \(g(x) = g(x + p_g)\) for all \(x \in \mathbf{R}\). We need to determine if the function \(h = f + g\) is periodic. For \(h\) to be periodic, there must exist a period \(p > 0\) such that \(h(x) = h(x + p)\) for all \(x\). However, the sum of two periodic functions is not necessarily periodic unless their periods are commensurate (i.e., there exists some positive integer multiples of \(p_f\) and \(p_g\) that are equal). In general, \(f(x + p) + g(x + p)\) might not equal \(f(x) + g(x)\) unless \(p\) is a common multiple of \(p_f\) and \(p_g\), which is not guaranteed for arbitrary periodic functions. Hence, the set is not closed under addition.

3. \textbf{Closed under scalar multiplication:} Suppose \(f\) is a periodic function with period \(p\) and \(\lambda\) is a scalar. The function \(\lambda f\) is periodic with the same period \(p\) because \(\lambda f(x + p) = \lambda f(x)\). Hence, the set is closed under scalar multiplication.

Since the set of periodic functions is not closed under addition, it fails to meet one of the necessary conditions for being a subspace. Therefore, the set of periodic functions from \(\mathbf{R}\) to \(\mathbf{R}\) is \textbf{not} a subspace of \(\mathbf{R}^{\mathbf{R}}\).

In conclusion, the composition (addition) of two periodic functions does not necessarily result in a new periodic function unless the functions have the same period or commensurate periods, so the set is not closed under addition.
\end{solution}


\part{Differential Equation}
\chapter{Ordinary Differential Equation}

\section{Typology of Differential Equation}

\section{Solution(s) of ODE}

\subsection{Solution Space of ODE}
\begin{theorem}[Solution Space of Homogeneous Linear Differential Equations]
Consider the n-th order linear homogeneous differential equation:
\[
a_n(x)\frac{d^ny}{dx^n} + a_{n-1}(x)\frac{d^{n-1}y}{dx^{n-1}} + \cdots + a_1(x)\frac{dy}{dx} + a_0(x)y = 0
\]
where $a_i(x)$ are continuous functions on an interval $I$ and $a_n(x) \neq 0$ for all $x \in I$. 
The set $S$ of all solutions to this equation forms a vector space over the field of real numbers.
\end{theorem}

\begin{proof}
To prove that $S$ is a vector space, we need to verify the eight vector space axioms:

1. Closure under addition: Let $y_1(x)$ and $y_2(x)$ be solutions to the equation. Then:
   \[
   a_n(x)\frac{d^n(y_1+y_2)}{dx^n} + \cdots + a_0(x)(y_1+y_2) = 
   \]
   \[
   \left(a_n(x)\frac{d^ny_1}{dx^n} + \cdots + a_0(x)y_1\right) + 
   \left(a_n(x)\frac{d^ny_2}{dx^n} + \cdots + a_0(x)y_2\right) = 0 + 0 = 0
   \]
   Thus, $y_1 + y_2$ is also a solution and belongs to $S$.

2. Closure under scalar multiplication: Let $y(x)$ be a solution to the equation and $c$ be any scalar. Then:
   \[
   a_n(x)\frac{d^n(cy)}{dx^n} + \cdots + a_0(x)(cy) = 
   c\left(a_n(x)\frac{d^ny}{dx^n} + \cdots + a_0(x)y\right) = c \cdot 0 = 0
   \]
   Thus, $cy$ is also a solution and belongs to $S$.

3. Commutativity of addition: For any $y_1, y_2 \in S$, clearly $y_1 + y_2 = y_2 + y_1$.

4. Associativity of addition: For any $y_1, y_2, y_3 \in S$, clearly $(y_1 + y_2) + y_3 = y_1 + (y_2 + y_3)$.

5. Additive identity: The zero function $y(x) = 0$ is a solution to the equation and acts as the additive identity.

6. Additive inverse: If $y(x)$ is a solution, then $-y(x)$ is also a solution, and $y + (-y) = 0$.

7. Scalar multiplication properties: For any scalars $c$ and $d$, and any $y, y_1, y_2 \in S$:
   \begin{itemize}
   \item $c(dy) = (cd)y$
   \item $1y = y$
   \item $(c+d)y = cy + dy$
   \item $c(y_1 + y_2) = cy_1 + cy_2$
   \end{itemize}
   These properties follow directly from the properties of real numbers and functions.

Therefore, the set $S$ of all solutions to the given linear homogeneous differential equation satisfies all the axioms of a vector space over the field of real numbers.
\end{proof}

\begin{theorem}[Solution Space of Non-Homogeneous Linear Differential Equations]
Consider the n-th order linear non-homogeneous differential equation:
\[
a_n(x)\frac{d^ny}{dx^n} + a_{n-1}(x)\frac{d^{n-1}y}{dx^{n-1}} + \cdots + a_1(x)\frac{dy}{dx} + a_0(x)y = f(x)
\]
where $a_i(x)$ and $f(x)$ are continuous functions on an interval $I$ and $a_n(x) \neq 0$ for all $x \in I$.
The set $S$ of all solutions to this equation forms an affine space over the vector space $V$ of solutions to the corresponding homogeneous equation.
\end{theorem}
\begin{proof}
To prove that $S$ is an affine space, we need to verify the following properties:

Existence of a vector space action: Let $V$ be the vector space of solutions to the corresponding homogeneous equation. We define an action $+: S \times V \to S$ as follows:
For any $y \in S$ and $v \in V$, $y + v$ is defined as the usual addition of functions.
Associativity of the action: For any $y \in S$ and $v, w \in V$:
\[
(y + v) + w = y + (v + w)
\]
This follows from the associativity of function addition.

Identity element: The zero function $0 \in V$ acts as the identity element:
\[
y + 0 = y \quad \text{for all } y \in S
\]
Uniqueness of displacement vector: For any $y_1, y_2 \in S$, there exists a unique $v \in V$ such that $y_1 + v = y_2$.

To prove these properties:

Well-definedness of the action:
Let $y \in S$ and $v \in V$. Then:
\[
a_n(x)\frac{d^n(y+v)}{dx^n} + \cdots + a_0(x)(y+v) =
\]
\[
\left(a_n(x)\frac{d^ny}{dx^n} + \cdots + a_0(x)y\right) +
\left(a_n(x)\frac{d^nv}{dx^n} + \cdots + a_0(x)v\right) = f(x) + 0 = f(x)
\]
Thus, $y + v \in S$.
Associativity of the action: This follows directly from the associativity of function addition.
Identity element: The zero function clearly satisfies this property.
Uniqueness of displacement vector:
Let $y_1, y_2 \in S$. Define $v = y_2 - y_1$. Then:
\[
a_n(x)\frac{d^nv}{dx^n} + \cdots + a_0(x)v =
\]
\[
\left(a_n(x)\frac{d^ny_2}{dx^n} + \cdots + a_0(x)y_2\right) -
\left(a_n(x)\frac{d^ny_1}{dx^n} + \cdots + a_0(x)y_1\right) = f(x) - f(x) = 0
\]
Thus, $v \in V$, and clearly $y_1 + v = y_2$. Uniqueness follows from the fact that if $w \in V$ also satisfies $y_1 + w = y_2$, then $v - w = 0$, so $v = w$.
Affine combination property:
For $y_1, y_2, \ldots, y_n \in S$ and scalars $\alpha_1, \alpha_2, \ldots, \alpha_n$ with $\sum_{i=1}^n \alpha_i = 1$:
\[
\sum_{i=1}^n \alpha_i y_i = y_1 + \sum_{i=2}^n \alpha_i (y_i - y_1)
\]
Since each $y_i - y_1 \in V$ and $V$ is a vector space, $\sum_{i=2}^n \alpha_i (y_i - y_1) \in V$.
Therefore, $\sum_{i=1}^n \alpha_i y_i \in S$.

Therefore, the set $S$ of all solutions to the given linear non-homogeneous differential equation satisfies all the properties of an affine space over the vector space $V$ of solutions to the corresponding homogeneous equation.
\end{proof}

\subsection{Solvable Conditions of ODEs}

In most cases, it is sure that an n-order linear ODE is solvable (has at least one solution).

\begin{theorem}[Existence of Solutions for Linear ODE]
Consider an \(n\)-th order linear differential equation:
\[
L(y) = a_n(x) \frac{d^n y}{dx^n} + a_{n-1}(x) \frac{d^{n-1} y}{dx^{n-1}} + \cdots + a_1(x) \frac{dy}{dx} + a_0(x) y = f(x),
\]
where \(f(x)\) is a given function and \(a_0(x), a_1(x), \dots, a_n(x)\) are the coefficient functions.

If the right-hand side \(f(x)\) is a continuous function in some interval \([a,b]\), and the coefficient functions \(a_0(x), a_1(x), \dots, a_n(x)\) are also continuous on this interval, then a solution to the differential equation \textit{exists} on this interval.

This implies that regardless of the specific form of \(f(x)\), as long as it is continuous (even only continuous in [$a,b$]), and the coefficients are continuous as well, we can guarantee the existence of a solution in a certain interval.
\end{theorem}
In general, for ordinary differential equations, \textit{continuity} is more fundamental than \textit{differentiability}. If \(f(x)\) and the coefficient functions are continuous, then a solution exists. If \(f(x)\) and the coefficients are differentiable, then the solution may be smoother (i.e., higher-order derivatives exist), but the existence of the solution only requires continuity.

We can show this with Existence-Uniqueness Theorem when discuss the existence of solutions to linear ODEs later.

Now we discuss the existence and uniqueness of the particular solution to linear ODEs. First, it is pretty clear that, without extra \textbf{valid} information provided as in IVP or BVP, we can only get a general solution that contains a constant term $C$, which can be taken as a subset of $\mathbf{C}^n(\R)$: the set of all real functions that are n-th order differentiable.

When it comes to the existence of a distinct, particular solution, we may refer to Existence and Uniqueness Theorem.

\begin{theorem}[Existence-Uniqueness Theorem(Picard-Lindelöf)]
Let $\frac{dy}{dx} = f(x, y)$
be an ODE subject to the initial condition \( y(x_0) = y_0 \).

If:

(a) \( f(x, y) \) is continuous for every \( (x, y) \in D \), where \( D \) is a rectangle bounded by the straight lines \( x = x_0 \pm a \) and \( y = y_0 \pm b \) in \( \mathbb{R} \).

(b) \( f(x, y) \) satisfies the Lipschitz condition of order 1, namely:
\[
|f(x, y_1) - f(x, y_2)| \leq K |y_1 - y_2|,
\]
where \( K \) is a constant dependent on \( D \), then there exists a unique solution \( y = \bar{y}(x) \) to the ODE such that \( y_0 = \bar{y}(x_0) \) for all \( x \in [x_0 - \delta, x_0 + \delta] \), where
\[
\delta < \min \left\{ a, \frac{b}{M}, \frac{1}{K} \right\} \quad \text{and} \quad M = \max_{(x, y) \in D} f(x, y).
\]
\end{theorem}
\begin{remark}
    The proof the the theorem is not compulsory to most people, and may take some time to show the technique applied (Picard iteration), but you may find it \href{https://en.wikipedia.org/wiki/Picard%E2%80%93Lindel%C3%B6f_theorem}{here} in case that you already have some basics of real analysis.
\end{remark}
     We also show how can we derive Lipschitz Continuity from the general definition to continuity in function. For consistency with the theorem, we will give the definition and discuss in binary functions.
\begin{definition}[Continuity]
    A function \( f(x, y) \) is continuous at \( (x_0, y_0) \) if for any \( \epsilon > 0 \), there exists \( \delta > 0 \) such that:
\[
|(x, y) - (x_0, y_0)| < \delta \quad \implies \quad |f(x, y) - f(x_0, y_0)| < \epsilon.
\]
This ensures small changes in \( (x, y) \) produce small changes in \( f(x, y) \).
\end{definition}
Lipschitz continuity is a stricter condition. A function is Lipschitz continuous in \( y \) if there exists \( K > 0 \) such that for any \( x \in D \) and all \( y_1, y_2 \in D \),
\[
|f(x, y_1) - f(x, y_2)| \leq K |y_1 - y_2|.
\]
While continuity guarantees no jumps, Lipschitz continuity controls the rate of change, bounding it by \( K \).

Lipschitz continuity implies continuity: given \( f(x, y) \) Lipschitz continuous, for any \( y_1, y_2 \),
\[
|f(x, y_1) - f(x, y_2)| \leq K |y_1 - y_2|.
\]
By setting \( \delta = \frac{\epsilon}{K} \), if \( |y_1 - y_2| < \delta \), then \( |f(x, y_1) - f(x, y_2)| < \epsilon \), which satisfies the continuity condition. Thus, every Lipschitz continuous function is continuous, but not every continuous function is Lipschitz, as general continuity lacks a bound on the rate of change.

There are more worth discussing on Lipschitz condition, since in real practice, we tend to use it's equivalent condition, which is easier to prove. 
\begin{corollary}
    For functions that are differentiable with respect to \( y \), the Lipschitz condition can be equivalently expressed in terms of the partial derivative. If the partial derivative of \( f(x, y) \) with respect to \( y \) is bounded, i.e.,
\[
\left| \frac{\partial f(x, y)}{\partial y} \right| \leq K \quad \text{for all } (x, y) \in D,
\]
then the function is Lipschitz continuous with respect to \( y \). In this case, the Lipschitz constant \( K \) is exactly the upper bound on the partial derivative.
\end{corollary}
\begin{proof}
To prove that bounded partial derivatives imply Lipschitz continuity, we proceed as follows:

\begin{enumerate}
    \item Assume that \( f(x, y) \) is continuously differentiable with respect to \( y \). Then for any two points \( y_1 \) and \( y_2 \), the fundamental theorem of calculus gives:
\[
f(x, y_1) - f(x, y_2) = \int_{y_2}^{y_1} \frac{\partial f(x, t)}{\partial y} \, dt
\]
    \item Taking the absolute value and applying the boundedness condition \( \left| \frac{\partial f(x, y)}{\partial y} \right| \leq K \), we have:
\[
|f(x, y_1) - f(x, y_2)| \leq \int_{y_2}^{y_1} \left| \frac{\partial f(x, t)}{\partial y} \right| \, dt \leq K |y_1 - y_2|
\]
\end{enumerate}

This confirms that if the partial derivative is bounded, the function satisfies the Lipschitz condition.
\end{proof}
 Thus, Lipschitz continuity can be viewed as an extension of the concept of bounded derivatives, guaranteeing controlled changes in the function's output as \( y \) varies.



\section{Separable ODEs}

\section{First-order linear homogeneous ODEs}

\section{First-order linear non-homogeneous ODEs}

\begin{theorem}[Solution of First-Order Linear Non-Homogeneous ODE]
Consider the first-order linear non-homogeneous ordinary differential equation:
\[
\frac{dy}{dx} + p(x)y = q(x)
\]
where $p(x)$ and $q(x)$ are continuous functions on an interval $I$. The general solution to this equation is given by:
\[
y(x) = \frac{1}{I(x)}\left[\int I(x)q(x)dx + C\right]
\]
where $I(x) = e^{\int p(x)dx}$ is an integrating factor, and $C$ is an arbitrary constant.
\end{theorem}

\begin{proof}
To solve this differential equation, we employ the method of integrating factors. Our goal is to find a function $I(x)$ such that when we multiply both sides of the equation by $I(x)$, the left-hand side becomes the derivative of a product.

We begin by multiplying both sides of the equation by an as-yet-undetermined function $I(x)$:

\[
I(x)\frac{dy}{dx} + I(x)p(x)y = I(x)q(x)
\]

Now, we aim to choose $I(x)$ such that the left-hand side is the derivative of the product $I(x)y$. Expanding this derivative using the product rule, we get:

\[
\frac{d}{dx}(I(x)y) = I(x)\frac{dy}{dx} + y\frac{dI}{dx}
\]

Comparing this with the left-hand side of our equation, we see that we need:

\[
y\frac{dI}{dx} = I(x)p(x)y
\]

This equality should hold for all $y$, so we can conclude:

\[
\frac{dI}{dx} = I(x)p(x)
\]

This is a separable differential equation for $I(x)$. Solving it, we find:

\[
I(x) = e^{\int p(x)dx}
\]

This is our integrating factor. With this choice of $I(x)$, our original equation becomes:

\[
\frac{d}{dx}(I(x)y) = I(x)q(x)
\]

We can now integrate both sides:

\[
I(x)y = \int I(x)q(x)dx + C
\]

where $C$ is an arbitrary constant of integration. Finally, solving for $y$, we obtain our general solution:

\[
y(x) = \frac{1}{I(x)}\left[\int I(x)q(x)dx + C\right]
\]

This completes our proof.
\end{proof}

\section{Second-order linear homogeneous ODEs}

\section{Second-order linear non-homogeneous ODEs}


%\stopcontents[part] % Manually stop the 'part' table of contents here so the previous Part page table of contents doesn't list the following chapters


\end{document}
